\ifx\allfiles\undefined
\input{../config/config}
\begin{document}
	% \input{../config/cover} 
	\else
	\fi
	%  ############################ 正文部分
\chapter{$Differentiation \,\, and \,\, Integration$}
\paragraph{\textbf{Motivation}}
	在\textbf{Riemann积分}的框架下,我们知道\textbf{积分}和\textbf{微分}可以视作一对互逆的运算. 而在这一章,我们将在全新的\textbf{Lebesgue测度}的框架下重新审视\textbf{积分}和\textbf{微分}之间的关系.
	
	\vspace{1em}
	下面先来描述一下想要解决的问题.
	\begin{itemize}
		\item Let $f \in \mathcal{L}^{1}(\R^d)$. 对于\textbf{变上限积分}$F(x) = \int_{a}^{x}{f(y) dy}$,我们知道根据\textbf{Riemann积分下的微积分基本定理},对$F$求导就会回到被积函数$f$ 本身. 那么我们就会好奇:
		
		\vspace{1em}
		
		\begin{itemize}
			\item 在\textbf{Lebesgue积分}的框架下,这个结论是否还成立?
			
			\item 如果成立的话,又对哪些$x$成立呢?
		\end{itemize}
	
		\vspace{1em}
		
		此时回顾求导的定义,即对于差商 (此处改写为更具一般性的符号$I = (x , x + h)$)
		\begin{align}
			\frac{F(x + h) - F(x)}{h} 
			= \frac{1}{h} \int_{x}^{x + h}{f(y) dy} 
			= \frac{1}{\left| I \right|} \int_{I}{f(y) dy}
		\end{align}
		对差商中的增量$h \to 0$,即得到\textbf{导数}的定义. 那么我们的问题就转化为了
		\begin{align}
			\lim_{\substack{\left| I \right| \to 0 \\ I \ni x}}{\frac{1}{\left| I \right|} \int_{I}{f(y) dy}}
			= f(x) 
			\,\,\,\, \text{holds for which $x$?}
		\end{align}
	
		\vspace{1em}
		
		更一般地,将上述问题从一维实直线$\R$ 推广至$\R^d$ 空间上,将区间$I$ 用开球$B$ 替换,得
		\begin{align}
			\lim_{\substack{m(B) \to 0 \\ B \ni x}}{\frac{1}{m(B)} \int_{B}{f(y) dy}}
			= f(x)
			\,\,\,\, \text{holds for which $x$?}
		\end{align}
	
		\vspace{1em}
		
		\begin{rmk}
			\begin{itemize}
				\item 此处看似是随着开球$B$的测度减小,$x \in B$ 在跟着$B$ ``跑",但实际上则相反:
				\begin{center}
					对于每个固定的$x$,让包含着$x$ 的球$B \ni x$ 不断减小其测度,最后取极限 
				\end{center}	
				而这也就是此处极限条件写为``$B \ni x$" 而非``$x \in B$" 的原因,逻辑更清晰.
				
				\vspace{1em}
				
				\item 事实上该结论对于\textbf{几乎处处的$x$} 都成立 (若$f$ \textbf{Lebesgue可积}),这就是后面要讲的\textbf{Lebesgue微分定理}.
			\end{itemize}
		\end{rmk}
	\end{itemize}
	
\newpage
\section{$Hardy-Littlewood$ 极大函数}
\paragraph{定义}
	下面我们给出\textbf{Hardy-Littlewood 极大函数}的定义.
	\begin{defn}\label{def 4.1.1}
		If $f \in \mathcal{L}^{1}(\R^d)$, we define its \underline{\textcolor{blue}{\textbf{maximal function}}} $Mf$ by
		\begin{align}
			Mf(x) = \sup_{B \ni x}{\frac{1}{m(B)} \int_{B}{\left| f(y) \right| dy}}
		\end{align}
	
		\vspace{1em}
		\begin{rmk}
			我们目前并不知道球面测度的具体数值与计算方法,但事实上我们也并不需要知道其具体数值,具体表现在: \\
			设$\R^d$ 中单位球$B(0 , 1)$ 的测度为$m(B(0 , 1)) = v_d$. $\forall B(x , r) \subset \R^d$,根据\textbf{Lebesgue测度的平移不变性和伸缩变换公式 (Prop \ref{prop 3.7.1})}
			\begin{center}
				$B(0 , r) = r B(0 , 1) \,\, \Rightarrow \,\, m(B(x , r)) = m(B(0 , r)) = r^d m(B(0 , 1)) = r^d v_d$
			\end{center}
		\end{rmk}
	\end{defn}

\vspace{2em}
\paragraph{性质}
	下面来说明\textbf{Hardy-Littlewood 极大函数}的三条\textbf{性质}.
	\begin{proposition}\label{prop 4.1.1}
		Suppose $f \in \mathcal{L}^{1}(\R^d)$. Then:
		\begin{enumerate}
			\item[(\rmnum{1})]$Mf$ is measurable.
			
			\item[(\rmnum{2})]$Mf(x) < \infty$ for a.e. $x$.
			
			\item[(\rmnum{3})]\textbf{weak-type inequality}. \\
			$Mf$ satisfies
			\begin{align}
				m\left( \left\{ x \in \R^d \mid Mf(x) > \alpha \right\} \right) 
				\leq \frac{A}{\alpha} \,\, \Vert f \Vert_{\mathcal{L}^1} , \,\, \forall \alpha > 0
			\end{align}
			where $A = 3^d$.
		\end{enumerate}
	
		\vspace{4em}
		\begin{proof}
			\begin{enumerate}
				\item [(\rmnum{1})]Let $E_\alpha = \{ x \in \R^d \mid Mf(x) > \alpha \}$. 下面证明$E_\alpha$ open. \\
				$\forall x \in E_\alpha$, by the \textbf{definition of $Mf$ (Def \ref{def 4.1.1})}, $\exists B_x \ni x$, $\st$
				\begin{align}
					\frac{1}{m(B_x)} \int_{B_x}{\left| f \right|} > \alpha
				\end{align}
				Then $\forall y \in B_x$, $B_x$ is also an open ball containing $y$, so we have $y \in E_\alpha$. i.e. $B_x \subset E_\alpha$. \\
				Therefore, $E_\alpha$ is open, specifically measurable for all $\alpha$. Then $Mf$ is measurable.
				
				\newpage
				
				\item[(\rmnum{2})]下面说明(\rmnum{3}) $\Rightarrow$ (\rmnum{2}): \\
				Let $E_\alpha = \{ x \in \R^d \mid Mf(x) > \alpha \}$. Then $E_n \searrow E = \{ x \in \R^d \mid Mf(x) = \infty \}$. \\
				Since $f \in \mathcal{L}^{1}(\R^d)$, $\Vert f \Vert_{\mathcal{L}^1}$ is finite. Then by \textbf{(\rmnum{3})}, $m(E_1) < \infty$. \\
				Then by \textbf{Thm \ref{thm 1.3.3}},
				\begin{align}
					m(E) = \lim_{n \to \infty}{m(E_n)} \leq \lim_{n \to \infty}{\frac{A}{n} \,\, \Vert f \Vert_{\mathcal{L}^1}} = 0
				\end{align}
				Therefore $m(E) = 0$. i.e. $Mf(x) < \infty$ for a.e. $x$.
				
				\vspace{6em}
				
				\item[(\rmnum{3})]在证明(\rmnum{3})之前,先来介绍\textbf{Vitali覆盖引理}.
				
				\vspace{2em}
				
				\begin{lemma}\label{lemma 4.1.1}
					\textbf{Vitali Covering Lemma (Elementary Version)}. \\
					Suppose $\mathcal{B} = \{ B_1 , B_2 , \cdots , B_N \}$, $B_i \subset \R^d$ are open balls, then there is a disjoint subcollection $B_{i_1} , \cdots , B_{i_k}$ that satisfies
					\begin{align}
						m\left( \bigcup_{l = 1}^{N}{B_l} \right) 
						\leq 3^d \sum_{j = 1}^{k}{m(B_{i_j})}
					\end{align}
					
					\vspace{1em}
					\begin{rmk}
						这是\textbf{Vitali覆盖引理}的\textbf{初等版本 (有限版本)},更一般的版本是对\textbf{一列}球结论成立.
					\end{rmk}
				
					\vspace{1em}
					\begin{proof}
						详见视频\href{https://www.bilibili.com/video/BV1FT411C7wM?p=31}{(非球心)Hardy-Littlewood极大函数} 23:10 (类似\textbf{贪心算法}的迭代步骤)
					\end{proof}
					
					\begin{figure}[thbp!]
						\centering
						\includegraphics[width=0.25\linewidth]{figure/4.1.1-1}
						\caption{Lemma 4.1.1}
						\label{pic : 4.1.1-1} % 添加图像引用标签
					\end{figure}
				\end{lemma}
			
				\newpage
				
				下面继续来证明(\rmnum{3}): \\
				Fix $\alpha > 0$, $\forall x \in B_\alpha$, $\exists$ open ball $B_x$, $\st$
				\begin{align}
					\frac{1}{m(B_x)} \int_{B_x}{\left| f \right|} > \alpha
				\end{align}
				So we have $E_\alpha \subset \underset{x \in E_\alpha}{\bigcup}{B_x}$. \\
				Since $E_\alpha$ is measurable (by \textbf{(\rmnum{1})}), then by \textbf{Thm \ref{thm 1.3.4} (Lebesgue测度的内正则性)}, \\
				$\forall \epsilon > 0$, $\exists$ compact $K_\epsilon \subset E_\alpha$, $\st$
				\begin{center}
					$m(E_\alpha \backslash K_\epsilon) \leq \epsilon$
				\end{center}
				i.e.
				\begin{center}
					$m(E_\alpha) - m(K_\epsilon) \leq \epsilon$
				\end{center}
				Since $K_\epsilon$ is compact, $K_\epsilon \subset \underset{x \in K_\epsilon}{\bigcup}{B_x}$, there exists a subcollection $B_{x_1} , \cdots , B_{x_N}$, $\st$
				\begin{center}
					$K_\epsilon \subset \overset{N}{\underset{l = 1}{\bigcup}}{B_{x_l}}$
				\end{center}
				Then by \textbf{Vitali Covering Lemma (Lemma \ref{lemma 4.1.1})}, there exists a subcollection $B_{x_{i_1}} , \cdots , B_{x_{i_k}}$, $\st$
				\begin{center}
					$m\left( \overset{N}{\underset{l = 1}{\bigcup}}{B_{x_l}} \right) 
					\leq 3^d \overset{k}{\underset{j = 1}{\sum}}{m(B_{x_{i_j}})}$
				\end{center}
				Therefore
				\begin{align}
					m(K_\epsilon) 
					\leq m\left( \bigcup_{l = 1}^{N}{B_{x_l}} \right)
					&\leq 3^d \sum_{j = 1}^{k}{m(B_{x_{i_j}})} \\
					&= \frac{3^d}{\alpha} \sum_{j = 1}^{k}{\alpha \cdot m(B_{x_{i_j}})} \\
					&\leq \frac{3^d}{\alpha} \int_{\bigcup_{j = 1}^{k}{B_{x_{i_j}}}}{\left| f \right|} \\
					&\leq \frac{3^d}{\alpha} \int_{\R^d}{\left| f \right|} \\
					&= \frac{3^d}{\alpha} \,\, \Vert f \Vert_{\mathcal{L}^1}
				\end{align}
				Then
				\begin{align}
					m(E_\alpha) 
					\leq m(K_\epsilon) + \epsilon 
					\leq \frac{A}{\alpha} \,\, \Vert f \Vert_{\mathcal{L}^1} + \epsilon
				\end{align}
				where $A = 3^d$, $\epsilon > 0$. \\
				Since $\epsilon$ is arbitrary, let $\epsilon \to 0$, we have
				\begin{align}
					m(E_\alpha) \leq \frac{A}{\alpha} \,\, \Vert f \Vert_{\mathcal{L}^1} , \,\, A = 3^d , \,\, \forall \alpha > 0
				\end{align}
			\end{enumerate}
		\end{proof}
	\end{proposition}

\newpage
\section{$Lebesgue$ 微分定理}
	\begin{center}
		在这一节我们将利用\textbf{Hardy-Littlewood极大函数}来证明\textbf{Lebesgue微分定理}.
	\end{center}
	
\subsection{$Chebyshev's \,\, Inequality$}
	在此之前,我们先来证明一个非常有用的不等式,即\textbf{切比雪夫不等式}.
	\begin{thm}\label{thm 4.2.1}
		\textbf{Chebyshev's Inequality}. \\
		If $g \in \mathcal{L}^{1}(\R^d)$, then
		\begin{align}
			m\left( \left\{ x \in \R^d \mid \left| g(x) \right| > \alpha \right\} \right)
			\leq \frac{1}{\alpha} \,\, \Vert g \Vert_{\mathcal{L}^1} , \,\, \forall \alpha > 0
		\end{align}
		
		\vspace{4em}
		\begin{proof}
			Let $E_\alpha = \{ x \in \R^d \mid \left| g(x) \right| > \alpha \}$. Then
			\begin{align}
				\Vert g \Vert_{\mathcal{L}^1}
				= \int_{\R^d}{\left| g \right|}
				\geq \int_{E_\alpha}{\left| g \right|}
				\geq \int_{E_\alpha}{\alpha}
				= \alpha \cdot m(E_\alpha)
			\end{align}
		\end{proof}
	\end{thm}

\newpage
\subsection{$The \,\, Lebesgue \,\, Differentiation \,\, Theorem$}
	下面我们就来给出\textbf{Lebesgue微分定理}.
	\begin{thm}\label{thm 4.2.2}
		If $f \in \mathcal{L}^{1}(\R^d)$, then
		\begin{align}
			\lim_{\substack{m(B) \to 0 \\ B \ni x}}{\frac{1}{m(B)} \int_{B}{f(y) dy}} = f(x) \,\,\,\, for \,\, a.e. \,\, x
		\end{align}
		
		\vspace{2em}
		\begin{rmk}
			\begin{itemize}
				\item \textbf{Lebesgue微分定理}说明了对于\textbf{几乎处处的$x$},当包含$x$ 的球体$B$ 的测度趋于0时,$f$ 在球体$B$ 上积分的平均值就会收敛到$f(x)$.
				
				\vspace{2em}
				
				\item 定理左侧实际上是关于集合$B$ 的函数的一个极限过程,用$\epsilon-\delta$语言叙述如下:\\
				$\forall \epsilon > 0$, $\exists \delta> 0$, $\st$ for all $B \ni x$ and $m(B) < \delta$, we have
				\begin{align}
					\left| \frac{1}{m(B)} \int_{B}{f(y) dy} - f(x) \right| \leq \epsilon
				\end{align}
			
				\vspace{2em}
				
				\item 要证明该定理,首先需要说明等式左侧\textbf{极限的存在性},但这并不好说明. 为了跳过说明其存在性的问题,我们需要引入类似\textbf{``上极限"}的函数,即: \\
				If suffices to show
				\begin{align}
					\lim_{\delta \to 0}{\sup_{\substack{m(B) < \delta \\ B \ni x}}{\left| \frac{1}{m(B)} \int_{B}{f(y) dy} - f(x) \right|}} = 0 \,\,\,\, for \,\, a.e. \,\, x
				\end{align}
				由于极限内的函数随着$\delta$ 递减而单调递减,又存在下界$0$,因此在$\delta = 0$ 处必存在右极限. 这样就跳过了原极限是否存在的问题.
				
				\vspace{2em}
				
				\item 事实上此处极限\textbf{``怪异"}的本质原因在于开球$B$ 的选取的任意性,若将其定义为以$x$ 为球心, $r$ 为半径的球,则可直接令$r \to 0$ 变为正常的函数极限,即
				\begin{align}
					\lim_{r \to 0}{\frac{1}{m(B(x , r))} \int_{B(x , r)}{f(y) dy}} = f(x)
				\end{align}
				在下一节我们会从\textbf{Hardy-Littlewood极大函数}开始,以此方法重新说明\textbf{Lebesgue微分定理}.
			\end{itemize}
		\end{rmk}
	
		\newpage
		\begin{proof}
			Let 
			\begin{align}
				E_\alpha 
				= \left\{ x \in \R^d \mid \lim_{\delta \to 0}{\sup_{\substack{m(B) < \delta \\ B \ni x}}{\left| \frac{1}{m(B)} \int_{B}{f(y) dy} - f(x) \right| > 2 \alpha}} \right\}
			\end{align}
			Then we \textbf{WTS (want to show)}:
			\begin{center}
				$m(E_\alpha) = 0$, $\forall \alpha \geq 0$
			\end{center}
			
			\vspace{1em}
			
			Fix $\alpha \geq 0$. By \textbf{Thm \ref{thm 3.2.3}}, $C_{c}(\R^d)$ is dense in $\mathcal{L}^{1}(\R^d)$ (有紧支集的连续函数), then \\
			$\forall \epsilon > 0$, $\exists g \in C_{c}(\R^d)$, $\st$
			\begin{center}
				$\Vert f - g \Vert_{\mathcal{L}^1} < \epsilon$
			\end{center}
			Since $g$ is uniformly continuous, then $\exists \delta > 0$, $\st$
			\begin{align}
				\left| \frac{1}{m(B)} \int_{B}{g(y) dy} - g(x) \right| 
				\leq \frac{1}{m(B)} \int_{B}{\left| g(y) - g(x) \right| dy}
				< \frac{1}{m(B)} \int_{B}{\epsilon dy} 
				= \epsilon
			\end{align}
			for all $B \ni x$ and $m(B) < \delta$. 
			
			\vspace{2em}
			
			下面对$m(E_\alpha)$ 进行估计. $\forall x \in E_\alpha$,
			\begin{align}
				\left| \frac{1}{m(B)} \int_{B}{f(y) dy} - f(x) \right|
				\leq \left| \frac{1}{m(B)} \int_{B}{(f(y) - g(y)) dy} \right|
				+ \left| \frac{1}{m(B)} \int_{B}{g(y) dy} - g(x) \right|
				+ \left| g(x) - f(x) \right|
			\end{align}
			对上述不等式中的开球$B \ni x$ 取上确界$\sup$,得
			\begin{align}
				\sup_{B \ni x}{\left| \frac{1}{m(B)} \int_{B}{f(y) dy} - f(x) \right|}
				\leq \sup_{B \ni x}{\left| \frac{1}{m(B)} \int_{B}{(f(y) - g(y)) dy} \right|}
				+ \sup_{B \ni x}{\left| \frac{1}{m(B)} \int_{B}{g(y) dy} - g(x) \right|}
				+ \left| g(x) - f(x) \right|
			\end{align}
			再令$m(B) \to 0$,由于根据\textbf{式 (4.24)},
			\begin{align}
				\lim_{\delta \to 0}{\sup_{\substack{m(B) < \delta \\ B \ni x}}{\left| \frac{1}{m(B)} \int_{B}{g(y) dy} - g(x) \right|}} = 0
			\end{align}
			因此
			\begin{align}
				\lim_{\delta \to 0}{\sup_{\substack{m(B) < \delta \\ B \ni x}}{\left| \frac{1}{m(B)} \int_{B}{f(y) dy} - f(x) \right|}}
				\leq \textcolor{red}{\lim_{\delta \to 0}{\sup_{\substack{m(B) < \delta \\ B \ni x}}{\left| \frac{1}{m(B)} \int_{B}{(f(y) - g(y)) dy} \right|}}}
				+ 0
				+ \left| g(x) - f(x) \right|
			\end{align}
			下面对\textcolor{red}{\textbf{红色部分}}进行估计. 根据对$\delta$ 的单调性可知,
			\begin{align}
				\textcolor{red}{\lim_{\delta \to 0}{\sup_{\substack{m(B) < \delta \\ B \ni x}}{\left| \frac{1}{m(B)} \int_{B}{(f(y) - g(y)) dy} \right|}}}
				\leq \sup_{B \ni x}{\frac{1}{m(B)} \int_{B}{\left| f(y) - g(y) \right| dy}}
				= M(f - g)(x)
			\end{align}
			又因为对于$\forall x \in E_\alpha$,
			\begin{align}
				\lim_{\delta \to 0}{\sup_{\substack{m(B) < \delta \\ B \ni x}}{\left| \frac{1}{m(B)} \int_{B}{f(y) dy} - f(x) \right|}}
				> 2 \alpha
			\end{align}
			所以
			\begin{align}
				&M(f - g)(x) + \left| g(x) - f(x) \right| > 2\alpha \\
				&\Rightarrow M(f - g)(x) > \alpha \,\, or \,\, \left| g(x) - f(x) \right| > \alpha \\
				&\Rightarrow E_\alpha 
				\subset \textcolor{purple}{\left\{ x \in \R^d \mid M(f - g)(x) > \alpha \right\}}
				\cup \textcolor{orange}{\left\{ x \in \R^d \mid \left| g - f \right| > \alpha \right\}}
			\end{align}
			下面分别来估计the \textcolor{purple}{\textbf{purple}} one 和 the \textcolor{orange}{\textbf{orange}} one 的测度.
			
			\vspace{2em}
			
			\begin{itemize}
				\item 由于$\left| f - g \right| \in \mathcal{L}^{1}{\R^d}$,因此根据\textbf{Chebyshev's Inequality (Thm \ref{thm 4.2.1})},
				\begin{align}
					m\left( \textcolor{orange}{\left\{ x \in \R^d \mid \left| g - f \right| > \alpha \right\}} \right)
					\leq \frac{1}{\alpha} \,\, \Vert f - g \Vert_{\mathcal{L}^1}
				\end{align}
				
				\vspace{1em}
				
				\item 根据\textbf{Hardy-Littlewood极大函数的weak-type inequality (Prop \ref{prop 4.1.1} (\rmnum{3}))},
				\begin{align}
					m\left( \textcolor{purple}{\left\{ x \in \R^d \mid M(f - g)(x) > \alpha \right\}} \right)
					\leq \frac{A}{\alpha} \,\, \Vert f - g \Vert_{\mathcal{L}^1}
				\end{align}
			\end{itemize}
		
			\vspace{2em}
			
			从而根据$\Vert f - g \Vert_{\mathcal{L}^1} < \epsilon$,
			\begin{align}
				m(E_\alpha)
				&\leq m\left( \textcolor{purple}{\left\{ x \in \R^d \mid M(f - g)(x) > \alpha \right\}} \right)
				+ m\left( \textcolor{orange}{\left\{ x \in \R^d \mid \left| g - f \right| > \alpha \right\}} \right) \\
				&\leq \frac{A + 1}{\alpha} \,\, \Vert f - g \Vert_{\mathcal{L}^1} \\
				&< \frac{A + 1}{\alpha} \epsilon , \,\, \forall \alpha \geq 0
			\end{align}
			Since $\epsilon > 0$ is arbitrary, let $\epsilon \to 0$, we get
			\begin{align}
				m(E_\alpha) = 0 , \,\, \forall \alpha \geq 0
			\end{align}
		\end{proof}
	\end{thm}


	%  ############################
	\ifx\allfiles\undefined
\end{document}
\fi