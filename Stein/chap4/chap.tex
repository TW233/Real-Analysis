\ifx\allfiles\undefined
\input{../config/config}
\begin{document}
	% \input{../config/cover} 
	\else
	\fi
	%  ############################ 正文部分
\chapter{$Differentiation \,\, and \,\, Integration$}
\paragraph{\textbf{Motivation}}
	在\textbf{Riemann积分}的框架下,我们知道\textbf{积分}和\textbf{微分}可以视作一对互逆的运算. 而在这一章,我们将在全新的\textbf{Lebesgue测度}的框架下重新审视\textbf{积分}和\textbf{微分}之间的关系.
	
	\vspace{1em}
	下面先来描述一下想要解决的问题.
	\begin{itemize}
		\item Let $f \in \mathcal{L}^{1}(\R^d)$. 对于\textbf{变上限积分}$F(x) = \int_{a}^{x}{f(y) dy}$,我们知道根据\textbf{Riemann积分下的微积分基本定理},对$F$求导就会回到被积函数$f$ 本身. 那么我们就会好奇:
		
		\vspace{1em}
		
		\begin{itemize}
			\item 在\textbf{Lebesgue积分}的框架下,这个结论是否还成立?
			
			\item 如果成立的话,又对哪些$x$成立呢?
		\end{itemize}
	
		\vspace{1em}
		
		此时回顾求导的定义,即对于差商 (此处改写为更具一般性的符号$I = (x , x + h)$)
		\begin{align}
			\frac{F(x + h) - F(x)}{h} 
			= \frac{1}{h} \int_{x}^{x + h}{f(y) dy} 
			= \frac{1}{\left| I \right|} \int_{I}{f(y) dy}
		\end{align}
		对差商中的增量$h \to 0$,即得到\textbf{导数}的定义. 那么我们的问题就转化为了
		\begin{align}
			\lim_{\substack{\left| I \right| \to 0 \\ I \ni x}}{\frac{1}{\left| I \right|} \int_{I}{f(y) dy}}
			= f(x) 
			\,\,\,\, \text{holds for which $x$?}
		\end{align}
	
		\vspace{1em}
		
		更一般地,将上述问题从一维实直线$\R$ 推广至$\R^d$ 空间上,将区间$I$ 用开球$B$ 替换,得
		\begin{align}
			\lim_{\substack{m(B) \to 0 \\ B \ni x}}{\frac{1}{m(B)} \int_{B}{f(y) dy}}
			= f(x)
			\,\,\,\, \text{holds for which $x$?}
		\end{align}
	
		\vspace{1em}
		
		\begin{rmk}
			\begin{itemize}
				\item 此处看似是随着开球$B$的测度减小,$x \in B$ 在跟着$B$ ``跑",但实际上则相反:
				\begin{center}
					对于每个固定的$x$,让包含着$x$ 的球$B \ni x$ 不断减小其测度,最后取极限 
				\end{center}	
				而这也就是此处极限条件写为``$B \ni x$" 而非``$x \in B$" 的原因,逻辑更清晰.
				
				\vspace{1em}
				
				\item 事实上该结论对于\textbf{几乎处处的$x$} 都成立 (若$f$ \textbf{Lebesgue可积}),这就是后面要讲的\textbf{Lebesgue微分定理}.
			\end{itemize}
		\end{rmk}
	\end{itemize}
	
\newpage
\section{$Hardy-Littlewood$ 极大函数}



	%  ############################
	\ifx\allfiles\undefined
\end{document}
\fi