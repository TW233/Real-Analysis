\title{{\Huge{\textbf{$Real \,\, Analysis$}}}\\
		\Large{\textbf{$Measure \,\, Theory , \,\, Integration , \,\, \& \,\, Hilbert \,\, Spaces$}}\footnote{参考书籍:\\
			\hspace*{4em} \textbf{《$Real \,\, Analysis -- Measure \,\, Theroy, \,\, Integration, \,\, \& \,\, Hilbert \,\, Spaces$》--- $Elias \,\, M. \,\, Stein$} \\
			\hspace*{4em} \textbf{《$Real \,\, Analysis -- Modern \,\, Techniques \,\, and \,\, Their \,\, Applications$》--- $Gerald \,\, B. \,\, Folland$}}}
\author{$-TW-$}
\date{\today}
\maketitle                   % 在单独的标题页上生成一个标题

\thispagestyle{empty}        % 前言页面不使用页码
\begin{center}
	\Huge\textbf{序}
\end{center}


\vspace*{3em}
\begin{center}
	\large{\textbf{天道几何,万品流形先自守;}}\\
	
	\large{\textbf{变分无限,孤心测度有同伦。}}
\end{center}

\vspace*{3em}
\begin{flushright}
	\begin{tabular}{c}
		\today \\ \small{\textbf{长夜伴浪破晓梦,梦晓破浪伴夜长}}
	\end{tabular}
\end{flushright}


\newpage                      % 新的一页
\pagestyle{plain}             % 设置页眉和页脚的排版方式(plain:页眉是空的,页脚只包含一个居中的页码)
\setcounter{page}{1}          % 重新定义页码从第一页开始
\pagenumbering{Roman}         % 使用大写的罗马数字作为页码
\tableofcontents              % 生成目录

\newpage                      % 以下是正文
\pagestyle{plain}
\setcounter{page}{1}          % 使用阿拉伯数字作为页码
\pagenumbering{arabic}
\setcounter{chapter}{0}    % 设置 -1 可作为第零章绪论从第零章开始