\ifx\allfiles\undefined
\documentclass[12pt, a4paper,oneside, UTF8]{ctexbook}
\usepackage[dvipsnames]{xcolor}
\usepackage{amsmath}   % 数学公式
\usepackage{amsthm}    % 定理环境
\usepackage{amssymb}   % 更多公式符号
\usepackage{graphicx}  % 插图
%\usepackage{mathrsfs}  % 数学字体
%\usepackage{newtxtext,newtxmath}
%\usepackage{arev}
\usepackage{kmath,kerkis}
\usepackage{newtxtext}
\usepackage{bbm}
\usepackage{enumitem}  % 列表
\usepackage{geometry}  % 页面调整
%\usepackage{unicode-math}
\usepackage[colorlinks,linkcolor=black]{hyperref}

\usepackage{wrapfig}


\usepackage{ulem}	   % 用于更多的下划线格式,
					   % \uline{}下划线,\uuline{}双下划线,\uwave{}下划波浪线,\sout{}中间删除线,\xout{}斜删除线
					   % \dashuline{}下划虚线,\dotuline{}文字底部加点


\graphicspath{ {flg/},{../flg/}, {config/}, {../config/} }  % 配置图形文件检索目录
\linespread{1.5} % 行高

% 页码设置
\geometry{top=25.4mm,bottom=25.4mm,left=20mm,right=20mm,headheight=2.17cm,headsep=4mm,footskip=12mm}

% 设置列表环境的上下间距
\setenumerate[1]{itemsep=5pt,partopsep=0pt,parsep=\parskip,topsep=5pt}
\setitemize[1]{itemsep=5pt,partopsep=0pt,parsep=\parskip,topsep=5pt}
\setdescription{itemsep=5pt,partopsep=0pt,parsep=\parskip,topsep=5pt}

% 定理环境
% ########## 定理环境 start ####################################
\theoremstyle{definition}
\newtheorem{defn}{\indent 定义}[section]

\newtheorem{lemma}{\indent 引理}[section]    % 引理 定理 推论 准则 共用一个编号计数
\newtheorem{thm}[lemma]{\indent 定理}
\newtheorem{corollary}[lemma]{\indent 推论}
\newtheorem{criterion}[lemma]{\indent 准则}

\newtheorem{proposition}{\indent 命题}[section]
\newtheorem{example}{\indent \color{SeaGreen}{例}}[section] % 绿色文字的 例 ,不需要就去除\color{SeaGreen}{}
\newtheorem*{rmk}{\indent \color{red}{注}}

% 两种方式定义中文的 证明 和 解 的环境:
% 缺点:\qedhere 命令将会失效【技术有限,暂时无法解决】
\renewenvironment{proof}{\par\textbf{证明.}\;}{\qed\par}
\newenvironment{solution}{\par{\textbf{解.}}\;}{\qed\par}

% 缺点:\bf 是过时命令,可以用 textb f等替代,但编译会有关于字体的警告,不过不影响使用【技术有限,暂时无法解决】
%\renewcommand{\proofname}{\indent\bf 证明}
%\newenvironment{solution}{\begin{proof}[\indent\bf 解]}{\end{proof}}
% ######### 定理环境 end  #####################################

% ↓↓↓↓↓↓↓↓↓↓↓↓↓↓↓↓↓ 以下是自定义的命令  ↓↓↓↓↓↓↓↓↓↓↓↓↓↓↓↓

% 用于调整表格的高度  使用 \hline\xrowht{25pt}
\newcommand{\xrowht}[2][0]{\addstackgap[.5\dimexpr#2\relax]{\vphantom{#1}}}

% 表格环境内长内容换行
\newcommand{\tabincell}[2]{\begin{tabular}{@{}#1@{}}#2\end{tabular}}

% 使用\linespread{1.5} 之后 cases 环境的行高也会改变,重新定义一个 ca 环境可以自动控制 cases 环境行高
\newenvironment{ca}[1][1]{\linespread{#1} \selectfont \begin{cases}}{\end{cases}}
% 和上面一样
\newenvironment{vx}[1][1]{\linespread{#1} \selectfont \begin{vmatrix}}{\end{vmatrix}}

\def\d{\textup{d}} % 直立体 d 用于微分符号 dx
\def\R{\mathbb{R}} % 实数域
\def\N{\mathbb{N}} % 自然数域
\def\C{\mathbb{C}} % 复数域
\def\Z{\mathbb{Z}} % 整数环
\def\Q{\mathbb{Q}} % 有理数域
\newcommand{\bs}[1]{\boldsymbol{#1}}    % 加粗,常用于向量
\newcommand{\ora}[1]{\overrightarrow{#1}} % 向量

% 数学 平行 符号
\newcommand{\pll}{\kern 0.56em/\kern -0.8em /\kern 0.56em}

% 用于空行\myspace{1} 表示空一行 填 2 表示空两行  
\newcommand{\myspace}[1]{\par\vspace{#1\baselineskip}}

%s.t. 用\st就能打出s.t.
\DeclareMathOperator{\st}{s.t.}

%罗马数字 \rmnum{}是小写罗马数字, \Rmnum{}是大写罗马数字
\makeatletter
\newcommand{\rmnum}[1]{\romannumeral #1}
\newcommand{\Rmnum}[1]{\expandafter@slowromancap\romannumeral #1@}
\makeatother
\begin{document}
	% \title{{\Huge{\textbf{$Real \,\, Analysis$}}}\\
		\Large{\textbf{$Measure \,\, Theory , \,\, Integration , \,\, \& \,\, Hilbert \,\, Spaces$}}\footnote{参考书籍:\\
			\hspace*{4em} \textbf{《$Real \,\, Analysis -- Measure \,\, Theroy, \,\, Integration, \,\, \& \,\, Hilbert \,\, Spaces$》--- $Elias \,\, M. \,\, Stein$} \\
			\hspace*{4em} \textbf{《$Real \,\, Analysis -- Modern \,\, Techniques \,\, and \,\, Their \,\, Applications$》--- $Gerald \,\, B. \,\, Folland$}}}
\author{$-TW-$}
\date{\today}
\maketitle                   % 在单独的标题页上生成一个标题

\thispagestyle{empty}        % 前言页面不使用页码
\begin{center}
	\Huge\textbf{序}
\end{center}


\vspace*{3em}
\begin{center}
	\large{\textbf{天道几何,万品流形先自守;}}\\
	
	\large{\textbf{变分无限,孤心测度有同伦。}}
\end{center}

\vspace*{3em}
\begin{flushright}
	\begin{tabular}{c}
		\today \\ \small{\textbf{长夜伴浪破晓梦,梦晓破浪伴夜长}}
	\end{tabular}
\end{flushright}


\newpage                      % 新的一页
\pagestyle{plain}             % 设置页眉和页脚的排版方式(plain:页眉是空的,页脚只包含一个居中的页码)
\setcounter{page}{1}          % 重新定义页码从第一页开始
\pagenumbering{Roman}         % 使用大写的罗马数字作为页码
\tableofcontents              % 生成目录

\newpage                      % 以下是正文
\pagestyle{plain}
\setcounter{page}{1}          % 使用阿拉伯数字作为页码
\pagenumbering{arabic}
\setcounter{chapter}{0}    % 设置 -1 可作为第零章绪论从第零章开始 
	\else
	\fi
	%  ############################ 正文部分
\chapter{$Measure \,\, Theory$}
\section{$Preliminaries$}
	\begin{defn}\label{def 1.1.1}
		A (closed) \underline{\textbf{rectangle}} $R$ in $\R^d$ is given by of d one-dimensional closed and bounded intervals
		\begin{align}
			R = [a_1 , b_1] \times [a_2 , b_2] \times \cdots \times [a_d , b_d]
		\end{align}
		where $a_j \leq b_j$ are real numbers , $j = 1 , 2 , \cdots , d$. In other word , we have
		\begin{align}
			R = \{ (x_1 , \cdots , x_d) \in \R^d \mid a_j \leq x_j \leq b_j , \,\, \forall j = 1 \sim d \}
		\end{align}
		The \underline{\textbf{volume}} of $R$ is
		\begin{align}
			\left| R \right| = (b_1 - a_1)\cdots(b_d - a_d)
		\end{align}
	\end{defn}

	\vspace*{3em}
	An \textbf{open} rectangle is the product of open intervals , and \underline{\textbf{the interior of the rectangle $R$}} is
	\begin{align}
		(a_1 , b_1) \times (a_2 , b_2) \times \cdots \times (a_d , b_d)
	\end{align}
	Also , a \underline{\textbf{cube}} is a rectangle for which $b_1 - a_1 = \cdots = b_d - a_d$.
	
	\vspace*{3em}
	\begin{defn}\label{def 1.1.2}
		A union of rectangles is said to be \underline{\textbf{almost disjoint}} if the interiors of them are disjoint.
	\end{defn}

	\vspace*{3em}
	\begin{lemma}\label{lemma 1.1.1}
		If a rectangle is the almost disjoint union of finitely many rectangles , say $R = \overset{N}{\underset{k = 1}{\bigcup}}{R_k}$, then
		\begin{align}
			\left| R \right| = \sum_{k = 1}^{N}{\left| R_k \right|}
		\end{align}
	
		\begin{rmk}
			本质上即指的是对于方体的任意的\uwave{\textbf{垂直划分}}可转化为\uwave{\textbf{“十字形”划分}}.
		\end{rmk}
	\end{lemma}

\newpage

	\begin{lemma}\label{lemma 1.1.2}
		If $R , R_1 , \cdots , R_N$ are rectangles , and $R \subset \overset{U}{\underset{k = 1}{\bigcup}}{R_k}$ , then
		\begin{align}
			\left| R \right| \leq \sum_{k = 1}^{N}{\left| R_k \right|}
		\end{align}
		
		\begin{rmk}
			 此即对$Lemma$ \ref{lemma 1.1.1} 的slight modification , 即各方体之间不一定再为almost disjoint.
		\end{rmk}
	\end{lemma}

	\vspace*{3em}
	Now we can give a description of the strcture of open sets in terms of cubes. Begin with the case of $\R$.
	
	\vspace*{2em}
	\begin{thm}\label{thm 1.1.3}
		Every open subset $\mathcal{O}$ of $\R$ can be written uniquely as countable union of disjoint open intervals.
		
		\begin{proof}
			For each $x \in \mathcal{O}$ , let $I_x$ be the largest open interval containing $x$ and contained in $\mathcal{O}$.
			\begin{enumerate}
				\item[$Step \,\, 1 :$]Construct $I_x$:\\
				$\mathcal{O}$ is open $\Rightarrow$ $x$ is contained in some small open interval contained in $\mathcal{O}$.\\
				Let 
				\begin{align}
					a_x &= inf\{ a < x \mid (a , x) \subset \mathcal{O} \} \\
					b_x &= sup\{ b > x \mid (x , b) \subset \mathcal{O} \}
				\end{align}
				Let $I_x = (a_x , b_x)$ , then $\mathcal{O} = \underset{x \in \mathcal{O}}{\bigcup}{I_x}$.
				
				\item[$Step \,\, 2 :$]Suppose $I_x \cap I_y \neq \varnothing$.\\
				$I_x \cup I_y$ is an open interval $\st 
				\begin{cases}
					x \in I_x \cup I_y \\
					I_x \cup I_y \subset \mathcal{O}
				\end{cases}$\\
				Since $I_x$ is maximal , $I_x \cup I_y \subset I_x$. Similarly , $I_x \cup I_y \subset I_y$.\\
				$\Rightarrow \,\, I_x = I_y$ \\
				$\Rightarrow \,\, if \,\, I_x \neq I_y , \,\, then \,\, I_x \cap I_y = \varnothing.$\\
				$\Rightarrow Z = \{ I_x \}_{x \in \mathcal{O}}$ is a disjoint famliy of sets.
				
				\item[$Step \,\, 3 :$]Since every $I_x$ contains at least a $\alpha_x \in \Q$ , construct a map $f$
				\begin{align}
					f : Z &\longrightarrow \Q \\
					I_x &\longmapsto \alpha_x
				\end{align}
				$f$ is an injective. $\Rightarrow \,\, \{ I_x \}_{x \in \mathcal{O}}$ is countable. $\Rightarrow \,\, \mathcal{O} = \overset{\infty}{\underset{j = 1}{\bigcup}}{(a_j , b_j)}$.
			\end{enumerate}
		\end{proof}
	\end{thm}

	\begin{thm}\label{thm 1.1.4}
		Every open set $\mathcal{O}$ of $\R^d$ , $d \geq 1$ , can be written as a countable union of almost disjoint closed cubes.
		
		\vspace*{2em}
		\begin{proof}
			Let 
			\begin{align}
				\mathcal{Q}_k &\coloneqq grid \,\, of \,\, 2^{-k} \Z^d , \,\, k \geq 0 \\
				\underline{A}(\mathcal{O} , k) &\coloneqq \{ Q \in \mathcal{Q}_k \mid Q \subset \mathcal{O} \} \\
				\overline{A}(\mathcal{O} , k) &\coloneqq \{ Q \in \mathcal{Q}_k \mid Q \cap \mathcal{O} \neq \varnothing \}
			\end{align}
			Since $\forall Q \in \underline{A}(\mathcal{O} , k) , \,\, \exists q \in Q^{\circ} , \,\, \st q \in \Q^d , $\\
			According to the Axiom of Choice , $\exists$ the map $f_k : \underline{A}(\mathcal{O} , k) \longrightarrow \Q^d$ , which is an injection.\\
			Hence $\underline{A}(\mathcal{O} , k)$ is countable.\\
			Let
			\begin{align}
				\underline{A}(\mathcal{O}) \coloneqq \bigcup_{k = 1}^{\infty}{\left( \underline{A}(\mathcal{O} , k) \,\, \backslash \,\, \underline{A}(\mathcal{O} , k - 1) \right)} \cup \underline{A}(\mathcal{O} , 0)
			\end{align}
			Then $\underline{A}(\mathcal{O})$ is also countable. Similarly define $\overline{A}(\mathcal{O})$.\\
			$\forall x \in \mathcal{O}$ , let $\delta_x \coloneqq inf \{ \left| y - x \right| \mid y \notin \mathcal{O} \}$. Since $\mathcal{O}$ is open , $\Rightarrow \,\, \delta_x > 0$.
			\begin{align}
				&\exists N_x \in \N , \,\, \st 2^{-k}\sqrt{d} \leq \frac{\delta_x}{2} < \delta_x , \forall k \geq N_x \\
				&\Rightarrow \forall Q \in \overline{A}(\mathcal{O} , N_x) , \,\, \st \left| s - t \right| \leq 2^{-N_x}\sqrt{d} < \delta_x , \forall s , t \in Q \\
				&\Rightarrow Since \,\, \mathcal{O} \subset \overline{A}(\mathcal{O}) , \,\, \exists Q_x \in \overline{A}(\mathcal{O} , N_x) \subset \overline{A}(\mathcal{O}) , \,\, \st x \in Q_x  \\
				&\Rightarrow x \in Q_x \subset \mathcal{O} \\
				&\Rightarrow x \in Q_x \in \underline{A}(\mathcal{O} , N_x) \subset \underline{A}(\mathcal{O}) \\
				&\Rightarrow \mathcal{O} \subset \underline{A}(\mathcal{O})
			\end{align}
			Obviously $\underline{A}(\mathcal{O}) \subset \mathcal{O}$ , so 
			\begin{align}
				\mathcal{O} = \underline{A}(\mathcal{O}) = \bigcup_{k = 1}^{\infty}{\left( \underline{A}(\mathcal{O} , k) \,\, \backslash \,\, \underline{A}(\mathcal{O} , k - 1) \right)} \cup \underline{A}(\mathcal{O} , 0)
			\end{align}
			which is a countable union of almost disjoint closed cubes.
		\end{proof}
	\end{thm}

\newpage
\section{$The \,\, Exterior \,\, Measure$}
\paragraph{$Definition$}
	The exterior measure attempts to describe the volume of a set $E$ by approximating it from the outside.\\ Loosely speaking , the exterior measure $m_{*}$ assigns to \textbf{any subset of $\R^d$} a first notion of size.
	\begin{defn}\label{def 1.2.1}
		If $E$ is a subset of $\R^d$ , the \underline{\textbf{exterior measure}} of $E$ is
		\begin{align}
			m_{*}(E) \coloneqq \inf{ \left\{ \sum_{j = 1}^{\infty}{\left| Q_j \right| \,\, \Big| \,\, E \subset \bigcup_{j = 1}^{\infty}{Q_j} , \,\, Q_j \,\, is \,\, a \,\, closed \,\, cube } \right\} }
		\end{align}
	
		\begin{rmk}
			\begin{itemize}
				\item \textbf{Well definition:}  $\forall E \subset \R^d , \,\, E \subset \overset{\infty}{\underset{n = 1}{\bigcup}}{Q_n} , \,\, Q_n = [-n , n]^d \subset \R^d$ , which means $m_{*}$ can be defined on every subset of $\R^d$.
				
				\item It is immediate from the definition that:\\
				For every $\epsilon > 0$ , there exists a covering $E \subset \overset{\infty}{\underset{j = 1}{\bigcup}}{Q_j} , \,\, \st$
				\begin{align}
					\sum_{j = 1}^{\infty}{m_{*}(Q_j)} \leq m_{*}(E) + \epsilon
				\end{align}
				
				\item It is important to note that it would \textbf{not suffice} to allow \textbf{finite sums} in the definition of $m_{*}(E)$. If one considered only coverings of $E$ by finite unions of cubes , the quantity is \textbf{in general larger} than $m_{*}(E)$.\\
				(In fact , it is defined as the \textbf{outer Jordan content $J_{*}(E)$}.)
				\begin{example}\label{ex 1.2.1}
					Consider the set $\Q \cap [0 , 1]$.
					\begin{itemize}
						\item For the outer Jordan content , since it's obvious that $J_{*}(\overline{E}) = J_{*}(E) , \,\, \forall E \subset \R^d , $\\
						$J_{*}(\Q \cap [0 , 1]) = J_{*}(\overline{\Q \cap [0 , 1]}) = J_{*}([0 , 1]) = 1$
						
						\item For the exterior measure , since $\Q \cap [0 , 1]$ is countable , let $\Q \cap [0 , 1]=  \{ x_1 , x_2 , \cdots \}$.\\
						Since for all $\epsilon > 0$ ,
						\begin{align}
							\Q \cap [0 , 1] \subset \bigcup_{j = 1}^{\infty}{[x_j - \frac{\epsilon}{2^j} , x_j + \frac{\epsilon}{2^j}]}
						\end{align}
						Hence $m_{*}(\Q \cap [0 , 1]) \leq \epsilon$. For $\epsilon$ is arbitrary , $m_{*}(\Q \cap [0 , 1]) = 0$.
					\end{itemize}
				\end{example}
			\end{itemize}
		\end{rmk}
	\end{defn}

\newpage
\paragraph{$Examples$}
	Let's check that whether the exterior measure matches our intuitive idea of volume.
	\begin{enumerate}
		\item[$Example \,\, 1 .$]\textbf{The exterior measure of a point is zero.}
		\begin{proof}
			It's clear that a point is a cube with $a_j = b_j , \forall j = 1 \sim d$ and which covers itself.
		\end{proof}
	
		\item[$Example \,\, 2 .$]\textbf{The exterior measure of a closed cube is equal to its volume.}
		\begin{proof}
			\begin{itemize}
				\item Let $Q \subset \R^d$ be a closed cube. Since $Q \subset Q$ , $m_{*}(Q) \leq \left| Q \right|.$
				
				\item Suppose $Q \subset \overset{\infty}{\underset{j = 1}{\bigcup}}{Q_j}$ by closed cubes. For fixed $\epsilon > 0$ , $\forall j \in \N$ , choose an open cube $S_j$ ,
				\begin{align}
					\st
					\begin{cases}
						S_j \supset Q_j\\
						\left| S_j \right| = (1 + \epsilon) \left| Q_j \right|
					\end{cases}
				\end{align}
				Then $Q \subset \overset{\infty}{\underset{j = 1}{\bigcup}}{S_j}$. Since $Q$ is compact , $\exists S_1 , \cdots , S_n \in \{ S_j \}_{j = 1}^{\infty} , \,\, \st Q \subset \overset{n}{\underset{j = 1}{\bigcup}}{S_j}$.\\
				Therefore , according to Lemma \ref{lemma 1.1.2}
				\begin{align}
					\left| Q \right| \leq \sum_{j = 1}^{n}{\left| S_j \right|} = (1 + \epsilon)\sum_{j = 1}^{n}{\left| Q_j \right|} \leq (1 + \epsilon)\sum_{j = 1}^{\infty}{\left| Q_j \right|}
				\end{align}
				For $\epsilon > 0$ is arbitrary , we get
				\begin{align}
					\left| Q \right| &\leq \sum_{j = 1}^{\infty}{\left| Q_j \right|} \\
					\left| Q \right| &\leq \inf{\sum_{j = 1}^{\infty}{\left| Q_j \right|}} = m_{*}(Q)
				\end{align}
			\end{itemize}
		\end{proof}
	
		\item[$Example \,\, 3 .$]\textbf{If $Q$ is an open cube , then $m_{*}(Q) = \left| Q \right|.$}
		\begin{proof}
			\begin{itemize}
				\item Since $Q \subset \overline{Q}$ , $m_{*}(Q) \leq \left| \overline{Q} \right| = \left| Q \right|$.
				
				\item We note that for all closed cubes $Q_0$ contained in $Q$ , then $m_{*}(Q_0) = \left| Q_0 \right| \leq m_{*}(Q)$.\\
				For fixed $\epsilon > 0$ which is suffice small , choose a closed cube $Q_0$ contained in $Q$ with a volume $\left| Q_0 \right| = (1 - \epsilon)\left| Q \right|$ , then we have
				\begin{align}
					\left| Q_0 \right| = (1 - \epsilon)\left| Q \right| \leq m_{*}(Q)
				\end{align}
				For $\epsilon$ is arbitrary , $\left| Q \right| \leq m_{*}(Q)$.
			\end{itemize}
		\end{proof}
	
		\item[$Example \,\, 4 .$]\textbf{The exterior measure of a rectangle $R$ is equal to its volume.}
		
		\item[$Example \,\, 5 .$]\textbf{$m_{*}(\R^d) = \infty.$}
		\begin{proof}
			Since any covering of $\R^d$ is also a covering of any cube $Q \subset \R^d$ , $m_{*}(\R^d) \geq m_{*}(Q)$\\
			$\forall N > 0 , \,\, \exists Q \subset \R^d , \,\, \st \left| Q \right| > N$ , so $m_{*}(\R^d) = \infty.$
		\end{proof}
	\end{enumerate}

\vspace*{3em}
\paragraph{$Properties$}
	\begin{enumerate}
		\item[$Observation \,\, 1.$]$(Monotonicity)$ \\
		If $E_1 \subset E_2$ , then $m_{*}(E_1) \leq m_{*}(E_2)$.
		
		\vspace*{2em}
		\item[$Observation \,\, 2.$]$(Countable \,\, sub-additivity)$\\
		If $E \subset \overset{\infty}{\underset{j = 1}{\bigcup}}{E_j}$ , then $m_{*}(E) \leq \overset{\infty}{\underset{j = 1}{\sum}}{m_{*}(E_j)}$.
		
		\vspace*{2em}
		\begin{proof}
			For a fixed $\epsilon > 0$ , for all $E_j$ , there exists a covering $\{ Q_{j_k} \}_{k = 1}^{\infty} , \,\, E \subset \overset{\infty}{\underset{k = 1}{\bigcup}}{Q_{j_k}} , \,\, \st$
			\begin{align}
				\overset{\infty}{\underset{k = 1}{\sum}}{m_{*}(Q_{j_k})} \leq m_{*}(E_j) + \frac{\epsilon}{2^j}
			\end{align}
			Since $E \subset \overset{\infty}{\underset{j = 1}{\bigcup}}{E_j} \subset \overset{\infty}{\underset{j = 1}{\bigcup}}\overset{\infty}{\underset{k = 1}{\bigcup}}{Q_{j_k}}$ , $\overset{\infty}{\underset{j = 1}{\bigcup}}\overset{\infty}{\underset{k = 1}{\bigcup}}{Q_{j_k}}$ covers $E$ , then
			\begin{align}
				m_{*}(E) \leq \overset{\infty}{\underset{j = 1}{\sum}}\overset{\infty}{\underset{k = 1}{\sum}}{m_{*}(Q_{j_k})} \leq \overset{\infty}{\underset{j = 1}{\sum}}{m_{*}(E_j)} + \epsilon
			\end{align}
			Since $\epsilon$ is arbitrary , $m_{*}(E) \leq \overset{\infty}{\underset{j = 1}{\sum}}{m_{*}(E_j)}$
		\end{proof}
		
		\vspace*{2em}
		\item[$Observation \,\, 3.$]\label{observation 1.2.3}If $E \subset \R^d$ , then $m_{*}(E) = \inf{\{ m_{*}(\mathcal{O}) \mid E \subset \mathcal{O} , \,\, \mathcal{O} \,\, is \,\, an \,\, open \,\, set \}}$.\\
		
		\vspace*{2em}
		\begin{proof}
			\begin{itemize}
				\item By monotonicity , $m_{*}(E) \leq m_{*}(\mathcal{O})$ , for all $\mathcal{O}$ covers $E$. Then take the infimum.
				
				\item For a fixed $\epsilon > 0$ , $\exists \,\, covering \,\,E \subset \overset{\infty}{\underset{j = 1}{\bigcup}}{Q_{j}} , \,\, \st$
				\begin{align}
					\overset{\infty}{\underset{j = 1}{\sum}}{m_{*}(Q_j)} \leq m_{*}(E) + \frac{\epsilon}{2}
				\end{align}
				For all $Q_j$ , choose an open set $\widetilde{Q_j}$ containing $Q_j$ with a volume $\left| \widetilde{Q_j} \right| \leq \left| Q_j \right| + \frac{\epsilon}{2^{j + 1}}$.\\
				Let $\mathcal{O} = \overset{\infty}{\underset{j = 1}{\bigcup}}{\widetilde{Q_j}}$ , then by Observation 2 ,
				\begin{align}
					m_{*}(\mathcal{O}) \leq \sum_{j = 1}^{\infty}{m_{*}(\widetilde{Q_j})} = \sum_{j = 1}^{\infty}{\left| \widetilde{Q_j} \right|} \leq \sum_{j = 1}^{\infty}{\left| Q_j \right|} + \frac{\epsilon}{2} \leq m_{*}(E) + \epsilon
				\end{align}
				Since $\epsilon$ is arbitrary , $m_{*}(\mathcal{O}) \leq m_{*}(E)$ , so $\inf{m_{*}(\mathcal{O})} \leq m_{*}(E)$.
			\end{itemize}
		\end{proof}
	
		\vspace*{2em}
		\item[$Observation \,\, 4.$]\label{observation 1.2.4}If $E = E_1 \cup E_2$ , and $d(E_1 , E_2) > 0$ , then
		\begin{align}
			m_{*}(E) = m_{*}(E_1) + m_{*}(E_2)
		\end{align}
	
		\vspace*{2em}
		\begin{proof}
			For a fixed $\epsilon > 0$ , $\exists$ a covering $E \subset \overset{\infty}{\underset{j = 1}{\bigcup}}{Q_{j}} , \,\, \st$
			\begin{align}
				\sum_{j = 1}^{\infty}{m_{*}(Q_j)} \leq m_{*}(E) + \epsilon
			\end{align}
			Subdevide the cubes $Q_j$ and assume that $diam(Q_j) <= \frac{d(E_1 , E_2)}{3}$. Then each $Q_j$ can intersect at most one of the two sets $E_1$ or $E_2$. Devide $\{ Q_j \}_{j = 1}^{\infty}$ into two subsets $\{ Q_j \}_{j \in J_1} , \,\, \{ Q_j \}_{j \in J2} , \,\, \st$
			\begin{align}
				E_1 \subset \bigcup_{j \in J_1}{Q_j} , \,\, E_2 \subset \bigcup_{j \in J_2}{Q_j}
			\end{align}
			$J_1$ and $J_2$ are both countable. $J_1 \cap J_2 = \varnothing$. Then
			\begin{align}
				m_{*}(E_1) \leq \sum_{j \in J_1}{m_{*}(Q_j)} , \,\, m_{*}(E_2) \leq \sum_{j \in J_2}{m_{*}(Q_j)}
			\end{align}
			Therefore
			\begin{align}
				m_{*}(E_1) + m_{*}(E_2) \leq \sum_{j \in J_1}{m_{*}(Q_j)} + \sum_{j \in J_2}{m_{*}(Q_j)} \leq \sum_{j = 1}^{\infty}{m_{*}(Q_j)} \leq m_{*}(E) + \epsilon
			\end{align}
			Since $\epsilon$ is arbitrary , $m_{*}(E_1) + m_{*}(E_2) \leq m_{*}(E)$.
		\end{proof}
	
		\newpage
		\item[$Observation \,\, 5.$]If a set $E$ is the countable union of almost disjoint cubes $E = \overset{\infty}{\underset{j = 1}{\bigcup}}{Q_{j}}$ , then
		\begin{align}
			m_{*}(E) = \sum_{j = 1}^{\infty}{\left| Q_j \right|}
		\end{align}
	
		\vspace*{2em}
		\begin{proof}
			For a fixed $\epsilon > 0$ , for all $Q_j$ , choose a closed cube $\widetilde{Q_j}$ strictly contained in $Q_j$ with its volume $\left| \widetilde{Q_j} \right| \geq \left| Q_j \right| - \frac{\epsilon}{2^j}$. Then for every $N \in \N$ , the cubes $\widetilde{Q_1} , \cdots , \widetilde{Q_N}$ are disjoint with a finite distance from one another. By Observation 4 , 
			\begin{align}
				m_{*}(\bigcup_{j = 1}^{N}{\widetilde{Q_j}}) = \sum_{i = 1}^{N}{\left| \widetilde{Q_j} \right|} \geq \sum_{j = 1}^{N}{\left| Q_j \right|} - \epsilon
			\end{align}
			Since $\overset{\infty}{\underset{j = 1}{\bigcup}}{\widetilde{Q_{j}}} \subset E$ , we conclude that for every $N$
			\begin{align}
				m_{*}(E) \geq \sum_{j = 1}^{N}{\left| Q_j \right|} - \epsilon
			\end{align}
			Let $N \to \infty$ , we deduce
			\begin{align}
				m_{*}(E) \geq \sum_{j = 1}^{\infty}{\left| Q_j \right|} - \epsilon
			\end{align}
			Since $\epsilon$ is arbitrary , $\overset{\infty}{\underset{j = 1}{\sum}}{\left| Q_j \right|} \leq m_{*}(E)$.
		\end{proof}
	\end{enumerate}

\newpage
\section{$Measurable \,\, sets \,\, and \,\, the \,\, Lebesgue \,\, measure$}
\subsection{$Measurable \,\, sets$}
\paragraph{Definition}
	\begin{defn}
		A subset $E$ of $\R^d$ is \underline{\textcolor{blue}{\textbf{$(Lebesgue) \,\, measurable$}}}, if for any $\epsilon > 0$ there exists an open set $O$ with $E \subset O$ and $m_{*}(O \backslash E) \leq \epsilon$.
		
		\vspace{1em}
		
		If $E$ is measurable, we define its \underline{\textcolor{blue}{\textbf{$(Lebesgue) \,\, measurable$}}} $m(E)$ by $m(E) = m_{*}(E)$.
	\end{defn}

	\begin{rmk}
		\begin{itemize}
			\item 可用映射的观点来理解外测度$m_{*}$ 与测度$m$ 的关系($Folland$).即
			\begin{align}
				m_{*} : \mathcal{P}(\R^d) &\longrightarrow \overline{\R}_{+} = [0 , +\infty] \\
				m : \mathcal{M} &\longrightarrow \overline{\R}_{+} = [0 , +\infty] \\
				m &= m_{*} \Big|_{\mathcal{M}}
			\end{align}
			其中$\mathcal{M} \subset \mathcal{P}(\R^d)$ 为$\R^d$ 中所有$(Lebesgue) \,\, measurable \,\, sets$ 构成的集合.
			
			\vspace{1em}
			
			\item 类比于抽象代数中各代数结构的性质,比如群$(group)$ 对加法 / 乘法封闭,我们下面探讨集合族$\mathcal{M}$ 对于可数个集合的运算$(countable \,\, unions , \,\, countable \,\, intersections , \,\, complement)$ 是否封闭.即通过此引出代数结构\textcolor{blue}{$\sigma-algebra$}.
		\end{itemize}
	\end{rmk}

	\vspace{2em}
\paragraph{Properties}
	下面开始探讨$(Lebesgue) \,\, measure$ 的部分性质.
	\begin{enumerate}
		\item[Property 1.]Every open set in $\R^d$ is measurable.
		
		\vspace{2em}
		
		\item[Property 2.]If $m_{*}(E) = 0$, then $E$ is measurable.
		\vspace{1em}
		\begin{proof}
			By Observation 3 in $\S \ref{observation 1.2.3}$, for a fixed $\epsilon > 0$, $\exists E \subset O$ open, $\st$
			\begin{align}
				m_{*}(O) \leq m_{*}(E) + \epsilon = \epsilon
			\end{align}
			Since $O \backslash E \subset O$, then $m_{*}(O \backslash E) \leq m_{*}(O) \leq \epsilon$.
		\end{proof}
	
		\newpage
		\item[Property 3.]\label{property 1.3.3}Let $\{ E_j \}_{j = 1}^{\infty}$ be a family of measurable sets, then $\overset{\infty}{\underset{j = 1}{\bigcup}}{E_j}$ is measurable.
		\begin{rmk}
			即说明集合族$\mathcal{M}$ 对$countable \,\, unions$ 封闭.
		\end{rmk}
		\begin{proof}
			Since $E_j$ is measurable, for a fixed $\epsilon > 0$, $\exists E_j \subset O_j$ open, $\st$
			\begin{align}
				m_{*}(O_j \backslash E_j) \leq \frac{\epsilon}{2^j}
			\end{align}
			Let $O = \overset{\infty}{\underset{j = 1}{\bigcup}}{O_j} \underset{open}{\subset} \R^d$, then 
			\begin{align}
				O \backslash \bigcup_{j = 1}^{\infty}{E_j} 
				&= \left( \bigcup_{j = 1}^{\infty}{O_j} \right) \cap \left( \bigcap_{j = 1}^{\infty}{E_{j}^c} \right) \\
				&= \bigcup_{j = 1}^{\infty}{\left( O_j \cap \left( \bigcap_{k = 1}^{\infty}{E_{k}^c} \right) \right)} 
				\subset \bigcup_{j = 1}^{\infty}{\left( O_j \cap E_{j}^c \right)} 
				= \bigcup_{j = 1}^{\infty}{\left( O_j \backslash E_{j} \right)}
			\end{align}
			Therefore
			\begin{align}
				m_{*}\left( O \backslash \bigcup_{j = 1}^{\infty}{E_j} \right) \leq m_{*}\left( \bigcup_{j = 1}^{\infty}{\left( O_j \backslash E_{j} \right)} \right) \leq \sum_{j = 1}^{\infty}{m_{*}\left( O_j \backslash E_j \right)} \leq \epsilon
			\end{align}
			So $\overset{\infty}{\underset{j = 1}{\bigcup}}{E_j}$ is measurable.
		\end{proof}
	
		\vspace{2em}
		\item[Property 4.]Closed sets are measurable.\\
		为了证明该性质,先证明如下的分离定理.
		\begin{lemma}\label{lemma 1.3.1}
			If $F$ is closed, $K$ is compact, and $K \cap F = \varnothing$, then $d(F , K) >0$.
			
			\vspace{1em}
			\begin{proof}
				反证法.Suppose $d(F , K) = 0$, then for any fixed $n \in \N , \,\, \exists x_n \in F , y_n \in K , \,\, \st$
				\begin{align}
					\left| x_n - y_n \right| \leq \frac{1}{n}
				\end{align}
				Since $K$ is compact, $\{ y_n \}_{n = 1}^{\infty}$ is bounded. Then there exists a subsequence $\{ y_{n_k} \}_{k = 1}^{\infty}$, $\st$
				\begin{align}
					y_{n_k} \to y_0 \in K , \,\, as \,\, k \to \infty
				\end{align}
				Since $\left| x_{n_k} - y_{n_k} \right| \leq \frac{1}{n_k}$, then
				\begin{align}
					\left| x_{n_k} - y_0 \right| \leq \left| x_{n_k} - y_{n_k} \right| + \left| y_{n_k} - y_0 \right| \to 0 , \,\, as \,\, k \to \infty
				\end{align}
				So $x_{n_k} \to y_0 \in F$, $y_0 \in F \cap K \neq \varnothing$ 矛盾.
			\end{proof}
		\end{lemma}
	
		\newpage
		下面证明 Property 4.
		\begin{proof}
			\begin{itemize}
				\item Suppose $F$ is bounded, then $F$ is compact.\\
				By Observation 3 in $\S \ref{observation 1.2.3}$, for a fixed $\epsilon > 0$, $\exists F \subset O$ open, $\st$
				\begin{align}
					m_{*}(O) \leq m_{*}(F) + \epsilon
				\end{align}
				Since $F$ is closed, $O \backslash F = O \cap F^c$ is open. By Thm\ref{thm 1.1.4}, $\exists \{ Q_j \}_{j = 1}^{\infty} , \,\, \st$
				\begin{align}
					O \backslash F = \bigcup_{j = 1}^{\infty}{Q_j}
				\end{align}
				For a fixed $N \in \N$, let $K = \overset{N}{\underset{j = 1}{\bigcup}}{Q_j}$, then $K$ is compact. By Lemma\ref{lemma 1.3.1}, $d(K , F) > 0$.\\
				Since $K \cup F \subset O$, by Observation 4 in $\S\ref{observation 1.2.4}$,
				\begin{align}
					m_{*}(K) + m_{*}(F) = m_{*}(K \cup F) \leq m_{*}(O)
				\end{align}
				So for each fixed $N \in \N$,
				\begin{align}
					\sum_{j = 1}^{N}{\left| Q_j \right|} = m_{*}(K) \leq m_{*}(O) - m_{*}(F) \leq \epsilon
				\end{align}
				Let $N \to \infty$, we get
				\begin{align}
					m_{*}(O \backslash F) = \sum_{j = 1}^{\infty}{\left| Q_j \right|} \leq \epsilon
				\end{align}
				Therefore, $F$ is measurable.
				
				\vspace{1em}
				
				\item For the general situation, since $\R^d = \overset{\infty}{\underset{j = 1}{\bigcup}}{B_j}$, then
				\begin{align}
					F = F \cap \R^d = \bigcup_{j = 1}^{\infty}{\left( F \cap B_j \right)}
				\end{align}
				Since $B_k$ is compact and $F$ is closed, then $F \cap B_j$ is compact.\\
				Due to the previous proof, $F \cap B_j$ is measurable. By Property 3 in $\S\ref{property 1.3.3}$,
				\begin{align}
					F = \bigcup_{j = 1}^{\infty}{\left( F \cap B_j \right)} \,\, is \,\, measurable.
				\end{align}
			\end{itemize}
		\end{proof}
		
		\newpage
		\item[Property 5.]If $E$ is measurable, then $E^c$ is measurable.
		\begin{rmk}
			即说明集合族$\mathcal{M}$ 对集合的补运算$complement$ 封闭.
		\end{rmk}
		\begin{proof}
			Since $E$ is measurable, then for all fixed $n \in \N$, $\exists E \subset O_n$ open, $\st m_{*}(O_n \backslash E) \leq \frac{1}{n}$.\\
			Let $S = \overset{\infty}{\underset{j = 1}{\bigcup}}{O_{j}^c} \subset E^c$. Since $O_{j}^c$ is closed, $O_{j}^c$ is measurable. Then $S$ is measurable.\\
			Since
			\begin{align}
				E^c \backslash S = E^c \cap \left( \bigcap_{j = 1}^{\infty}{O_j} \right) = \bigcap_{j = 1}^{\infty}{\left( E^c \cap O_j \right)} \subset E^c \cap O_n = O_n \backslash E , \,\, \forall n \in \N
			\end{align}
			Then, $m_{*}(E^c \backslash S) \leq m_{*}(O_n \backslash E) \leq \frac{1}{n} , \,\, \forall n \in \N$. So $E^c \backslash S$ is measurable.\\
			Therefore, $E^c = \left( E^c \backslash S \right) \cup S$ is measurable.
		\end{proof}
	
		\vspace{2em}
		\item[Property 6.]If $\{ E_j \}_{j = 1}^{\infty}$ is a family of measurable sets, then $\overset{\infty}{\underset{j = 1}{\bigcap}}{E_j}$ is measurable.
		
		\begin{rmk}
			即说明集合族$\mathcal{M}$ 对$countable intersections$ 封闭.
		\end{rmk}
		
		\begin{proof}
			Since
			\begin{align}
				\bigcap_{j = 1}^{\infty}{E_j} = \left( \bigcup_{j = 1}^{\infty}{E_{j}^c} \right)^c
			\end{align}
			Then, $E_{j}^c$ is measurable and so $\overset{\infty}{\underset{j = 1}{\bigcap}}{E_j}$ is measurable.
		\end{proof}
	\end{enumerate}
	
	\vspace{2em}
	综上,本节介绍了$(Lebesgue) \,\, measurable \,\, sets$ 的性质,并且证明了$Lebesgue \,\, measurable \,\, sets$ 构成的集合族$\mathcal{M}$ 对$countable \,\, unions , \,\, countable \,\, intersections , \,\, complement$ 运算封闭. 从而$(\mathcal{M} , \cup , \cap , complement)$ 构成代数结构,即为后续介绍的\textcolor{blue}{$\sigma-algebra$}.
	
\newpage
\subsection{$Lebesgue \,\, measure$}
	下面着重来介绍一下$Lebesgue \,\, measure$ 的$properties$.
	
	\vspace{2em}
\paragraph{可数可加性}
	首先便是可数可加性$countable \,\, additivity$.
	\begin{thm}\label{thm 1.3.2}
		If $E_1 , E_2 , \cdots $ are disjoint measurable sets, then
		\begin{align}
			m(\bigcup_{j = 1}^{\infty}{E_j}) = \sum_{j = 1}^{\infty}{m(E_j)}
		\end{align}
	
		\vspace{2em}
		\begin{proof}
			Since $m(\overset{\infty}{\underset{j = 1}{\bigcup}{E_j}}) \leq \overset{\infty}{\underset{j = 1}{\sum}}{m(E_j)}$ always holds, we then proof the reverse inequality.
			\begin{itemize}
				\item Suppose that $E_j$ is bounded.\\
				Since $E_{j}^c$ is measurable, for any fixed $\epsilon > 0$, there exists an closed subset $F_j \subset E_j$, $\st$
				\begin{align}
					m(E_j \backslash F_j) \leq \frac{\epsilon}{2^j}
				\end{align}
				Since $E_j$ is bounded, $F_j$ is compact. \\
				Let $K = \overset{N}{\underset{j = 1}{\bigcup}}{F_j}$ be a disjoint union of compact sets for some fixed $N$, then
				\begin{align}
					K &\subset \bigcup_{j = 1}^{\infty}{E_j} \\
					m(K) &= \sum_{j = 1}^{N}{m(F_j)} \leq m(\bigcup_{j = 1}^{\infty}{E_j})
				\end{align}
				Since
				\begin{align}
					m(E_j) \leq m(E_j \backslash F_j) + m(F_j) \leq m(F_j) + \frac{\epsilon}{2^j}
				\end{align}
				Therefore
				\begin{align}
					\sum_{j = 1}^{N}{m(E_j)} - \epsilon \leq \sum_{j = 1}^{N}{m(F_j)} \leq m(\bigcup_{j = 1}^{\infty}{E_j})
				\end{align}
				Let $N \to \infty$, for $\epsilon$ is arbitrary, we get
				\begin{align}
					\sum_{j = 1}^{\infty}{m(E_j)} \leq m(\bigcup_{j = 1}^{\infty}{E_j})
				\end{align}
			
				\newpage
				
				\item In the general case, we choose the sequence of cubes $\{ Q_k \}_{k = 1}^{\infty}$, $Q_k = [-k , k]^d \subset \R^d$.\\
				Let $S_1 = Q_1$, $S_k = Q_k - Q_{k - 1}$, $\forall k \geq 2$. Then $\{ S_k \}_{k = 1}^{\infty}$ are disjoint and  bounded.\\
				Since $\{ S_k \}_{k = 1}^{\infty}$ covers $\R^d$,
				\begin{align}
					E_j &= \bigcup_{k = 1}^{\infty}{(E_j \cap S_k)} \\
					\bigcup_{j = 1}^{\infty}{E_j} &= \bigcup_{j = 1}^{\infty}\bigcup_{k = 1}^{\infty}{(E_j \cap S_k)}
				\end{align}
				Since $E_j \cap S_k$ is bounded and disjoint, by the previous case,
				\begin{align}
					m(\bigcup_{j = 1}^{\infty}{E_j}) = \sum_{j = 1}^{\infty}\sum_{k = 1}^{\infty}{m(E_j \cap S_k)} = \sum_{j = 1}^{\infty}{m(E_j)}
				\end{align}
			\end{itemize}
		\end{proof}
	\end{thm}
	
	\vspace{2em}
\paragraph{单调连续性}
	下面我们可以给出单调可测集合列的连续性.$continuity \,\, from \,\, below / above$
	\begin{thm}\label{thm 1.3.3}
		Let $E_1 , E_2 , \cdots$ be measurable sets in $\R^d$.
		\begin{enumerate}
			\item[(\rmnum{1})]If $E_k \nearrow E$, then $m(E) = \underset{n \to \infty}{\lim}{m(E_n)}$.
			
			\item[(\rmnum{2})]If $E_k \searrow E$ and $m(E_1) < \infty$, then $m(E) = \underset{n \to \infty}{\lim}{m(E_n)}$.
		\end{enumerate}
	
		\begin{rmk}
			\begin{itemize}
				\item 事实上即可写为
				\begin{align}
					m(\lim_{n \to \infty}{E_n}) = \lim_{n \to \infty}{m(E_n)}
				\end{align}
				即单调可测集合列\textbf{可交换极限与测度顺序}.
				
				\item (\rmnum{2})中条件$m(E_1)$ finite 不可省略,下面给出一个反例.
				\begin{example}
					If $E_n = (n , +\infty)$, then $m(E_n) = \infty$ and $E = \overset{\infty}{\underset{j = 1}{\cap}}{E_j} = \varnothing$. So
					\begin{align}
						m(E) = m(\lim_{n \to \infty}{E_j}) = 0 , \,\, \lim_{n \to \infty}{m(E_j)} = \infty
					\end{align}
				\end{example}
			\end{itemize}
			
		\end{rmk}
		
		\begin{proof}
			\begin{enumerate}
				\item[(\rmnum{1})]Let $S_1 = E_1$, $S_k = E_k - E_{k - 1}$, $\forall k \geq 2$. Then $\{ S_k \}_{k = 1}^{\infty}$ are disjoint and measurable.\\
				Since $E = \overset{\infty}{\underset{k = 1}{\bigcup}}{E_k} = \overset{\infty}{\underset{k = 1}{\bigcup}}{S_k}$, by Thm\ref{thm 1.3.2},
				\begin{align}
					m(E) = \sum_{k = 1}^{\infty}{m(S_k)} = \lim_{N \to \infty}{\sum_{k = 1}^{N}{m(S_k)}} = \lim_{N \to \infty}{m(\bigcup_{k = 1}^{N}{S_k})} = \lim_{N \to \infty}{m(E_N)}
				\end{align}
			
				\item[(\rmnum{2})]Let $S_1 = E_1$, $S_k = E_k - E_{k + 1}$, $\forall k \geq 2$. Then $\{ S_k \}_{k = 1}^{\infty}$ are disjoint and measurable.\\
				Since $E_1 = E \cup \left( \overset{\infty}{\underset{k = 1}{\bigcup}}{S_k} \right)$, then
				\begin{align}
					m(E_1) = m(E) + \sum_{k = 1}^{\infty}{m(S_k)} = m(E) + \lim_{N \to \infty}{m(\bigcup_{k = 1}^{N}{S_k})} = m(E) + \lim_{N \to \infty}{m(E_1 - E_N)}
				\end{align}
				For $E_1 = (E_1 - E_N) \sqcup E_N$ is a disjoint union,
				\begin{align}
					m(E_1 - E_N) = m(E_1) - m(E_N)
				\end{align}
				Thus
				\begin{align}
					m(E_1) &= m(E) + \lim_{N \to \infty}{m(E_1 - E_N)} = m(E) + m(E_1) - \lim_{N \to \infty}{m(E_N)} \\
					m(E) &= \lim_{N \to \infty}{m(E_N)}
				\end{align}
			\end{enumerate}
		\end{proof}
	\end{thm}

	\vspace{2em}
\paragraph{$Geometric \,\, insight \,\, of \,\, measurable \,\, sets$}

	最后我们来给出$(Lebesgue) \,\, measurable \,\, sets$ 的几何性质(与开集、闭集、紧集等之间的关系).
	\begin{thm}\label{thm 1.3.4}
		Suppose $E \subset \R^d$ is measurable, then $\forall \epsilon > 0 :$
		\begin{enumerate}
			\item[(\rmnum{1})]$\exists$ open $O \supset E$ with $m(O \backslash E) \leq \epsilon$.
			
			\item[(\rmnum{2})]$\exists$ closed $F \subset E$ with $m(E \backslash F) \leq \epsilon$.
			
			\item[(\rmnum{3})]If $m(E) < \infty$, $\exists$ compact $K \subset E$ with $m(E \backslash K) \leq \epsilon$.
			
			\item[(\rmnum{4})]If $m(E) < \infty$, $\exists F = \overset{N}{\underset{j = 1}{\bigcup}}{Q_j}$, $\{ Q_j \}_{j = 1}^\infty$ are closed cubes, $\st m(E \triangle F) \leq \epsilon$.
		\end{enumerate}
		
		\vspace{2em}
		\begin{proof}
			\begin{enumerate}
				\item[(\rmnum{1})]It's just the definition of measurability.
				
				\item[(\rmnum{2})]Since $E_{j}^c$ is measurable, $\exists$ open $O_j \supset E_{j}^c$, $\st$
				\begin{align}
					m(O_j \backslash E_{j}^c) \leq \epsilon
				\end{align}
				Since $O_{j}^c \subset E_j$ is closed and $E_j \backslash O_{j}^c = O_j \backslash E_{j}^c$, let $F = O_{j}^c$ closed, then
				\begin{align}
					m(E_j \backslash F) = m(O_j \backslash E_{j}^c) \leq \epsilon
				\end{align}
				
				\item[(\rmnum{3})]By (\rmnum{2}), $\exists$ closed $F \subset E$, $\st m(E \backslash F) \leq \frac{\epsilon}{2}$.\\
				Let $B_n$ denote the closed ball centered at the origin of radius n, then $B_n$ is compact.
				\begin{align}
					F = \bigcup_{j = 1}^{\infty}{(F \cap B_k)}
				\end{align}
				Let $K_n = \overset{n}{\underset{k = 1}{\bigcup}}{(F \cap B_k)}$, then $K_n$ is compact and $K_n \nearrow F \Rightarrow E \backslash K_n \nearrow E \backslash F$.\\
				Since $m(E \backslash K_1) \leq m(E)$ is finite, by Thm\ref{thm 1.3.3}(\rmnum{2})
				\begin{align}
					\lim_{n \to \infty}{m(E \backslash K_n)} = m(E \backslash F)
				\end{align}
				As for $\epsilon > 0$, $\exists N \in \N$, $\st$ for all $n \geq N$
				\begin{align}
					\left| m(E \backslash K_n) - m(E \backslash F) \right| &\leq \frac{\epsilon}{2} \\
					m(E \backslash K_n) &\leq m(E \backslash F) + \frac{\epsilon}{2} \leq \epsilon
				\end{align}
				Therefore, $m(E \backslash K_N) \leq \epsilon$, where $K_N \subset E$ is compact.
				
				\item[(\rmnum{4})]$\exists$ open $O \supset E$, $\st m(O \backslash E) \leq \frac{\epsilon}{2}$. By Thm\ref{thm 1.1.4}, $\exists \{ Q_j \}_{j = 1}^{\infty}$, $\st$
				\begin{align}
					E \subset O = \bigcup_{j = 1}^{\infty}{Q_j}
				\end{align}
				So
				\begin{align}
					m(O) = \sum_{j = 1}^{\infty}{\left| Q_j \right|} \leq m(O \backslash E) + m(E) \leq \frac{\epsilon}{2} + m(E)
				\end{align}
				Since $m(E)$ is finite, $\sum_{j = 1}^{\infty}{\left| Q_j \right|}$ converges. Then $\exists N \in \N$, $\st$
				\begin{align}
					\sum_{j = N + 1}^{\infty}{\left| Q_j \right|} \leq \frac{\epsilon}{2}
				\end{align}
				Let $F = \overset{N}{\underset{j = 1}{\bigcup}}{Q_j}$. Since $E \triangle F = (E \backslash F) \sqcup (F \cap E)$, then
				\begin{align}
					m(E \triangle F) 
					&= m(E \backslash F) + m(F \backslash E) \\
					&\leq m(\bigcup_{j = N + 1}^{\infty}{Q_j}) + m(\bigcup_{j = 1}^{\infty}{Q_j} \backslash E) \\ 
					&= \sum_{j = N + 1}^{\infty}{\left| Q_j \right|} + \sum_{j = 1}^{\infty}{\left| Q_j \right|} - m(E) \\
					&\leq \frac{\epsilon}{2} + \frac{\epsilon}{2} = \epsilon
				\end{align}
			\end{enumerate}
		\end{proof}
	\end{thm}

\newpage
\section{$\sigma-algebras \,\, and \,\, Borel \,\, sets$}
\subsection{$\sigma-algebra$}
	首先给出$\R^d$ 中\textbf{$algebra$} 的定义.
	\begin{defn}
		Let $\mathcal{A} \subset \mathcal{P}(\R^d)$. $\mathcal{A}$ is called an \underline{\textcolor{blue}{\textbf{$algebra$}}} if
		\begin{enumerate}
			\item[(1)]If $A_1 , \cdots , A_n \in \mathcal{A}$, then $\overset{n}{\underset{j = 1}{\bigcup}}{A_j} \in \mathcal{A}$.
			
			\item[(2)]If $A \in \mathcal{A}$, then $A^c \in \mathcal{A}$.
		\end{enumerate}
		
		\begin{rmk}
			容易证明,若$\mathcal{A}$ 为$\R^d$ 中$algebra$,则其对$finite \,\, intersections$ 也封闭,同时$\varnothing , \R^d \in \mathcal{A}$.
		\end{rmk}
	\end{defn}

	\vspace{2em}
	下面给出$\R^d$ 中$\sigma-algebra$ 的定义.(将$algebra$ 中的$finite$ 条件加强为$countable$)
	\begin{defn}
		Let $\mathcal{M} \subset \mathcal{P}(\R^d)$. $\mathcal{M}$ is a \underline{\textcolor{blue}{$\sigma-algebra$}} if
		\begin{enumerate}
			\item[(1)]If $A_1 , A_2 , \cdots \in \mathcal{M}$, then $\overset{\infty}{\underset{j = 1}{\bigcup}}{A_j} \in \mathcal{M}$.
			
			\item[(2)]If $A \in \mathcal{M}$, then $A^c \in \mathcal{M}$.
		\end{enumerate}
		
		\begin{rmk}
			容易证明$\mathcal{M}$ 对$countable \,\, intersections$ 同样封闭,$\varnothing , \R^d \in \mathcal{M}$.
		\end{rmk}
		
		\begin{example}
			All Lebesgue measurable sets forms a $\sigma-algebra$ $\mathcal{M}$.
		\end{example}
	\end{defn}

	\vspace{2em}
	类比线性空间、拓扑空间中(拓扑)基的概念,下面给出\textbf{生成$\sigma-algebra$}的概念.
	\begin{defn}
		Let $\mathcal{A} \subset \mathcal{P}(\R^d)$, then the $\sigma-algebra$ \underline{\textcolor{blue}{\textbf{generated by $\mathcal{A}$}}} is the smallest $\sigma-algebra$ containing $\mathcal{A}$.
		
		\begin{rmk}
			即为 the intersection of all $\sigma-algebras$ containing $\mathcal{A}$,这也说明了对于任一给定的集族$\mathcal{A}$,其生成的$\sigma-algebra$ 必存在且唯一.
		\end{rmk}
	\end{defn}

\newpage
\subsection{$Borel \,\, sets$}
	下面给出$Borel \,\, \sigma-algebra$ 及$Borel \,\, sets$ 的定义.
	\begin{defn}\label{def 1.4.4}
		The \underline{\textcolor{blue}{\textbf{$Borel \,\, \sigma-algebra$}}} is the $\sigma-algebra$ generated by all open sets in $\R^d$, \\
		denoted by \textcolor{blue}{$\mathcal{B}_{\R^d}$}.\\
		Elements of this $\sigma-algebra$ are called \underline{\textcolor{blue}{$Borel \,\, sets$}}.
		
		\begin{rmk}
			事实上,$Borel \,\, \sigma-algebra$ 为Lebesgue countable sets 的一个真子集,后续会利用Cantor集证明.
		\end{rmk}
	\end{defn}

	\vspace{2em}
	为了方便研究$Borel \,\, \sigma-algebra$ 的结构,我们把其中较为复杂(非平凡)的元素单独拎出来并称为$G_\delta , F_\sigma$.
	\begin{defn}\label{def 1.4.5}
		\begin{enumerate}
			\item The countable intersections of open sets are called \underline{\textcolor{blue}{$G_\delta$ sets}}.
			
			\item The countable unions of closed sets are called \underline{\textcolor{blue}{$F_\sigma$ sets}}.
		\end{enumerate}
	\end{defn}
	
	\vspace{2em}
	下面我们可给出$\mathcal{B}_{\R^d}$ 与Lebesgue可测集$\mathcal{L}$ 之间的关系.($\mathcal{L}$ 只比$\mathcal{B}_{\R^d}$ 多了一些零测集)
	\begin{thm}\label{thm 1.4.1}
		$E \subset \R^d$ is $\mathcal{L}-measurable$
		\begin{enumerate}
			\item[(\rmnum{1})]if and only if $E = G_\delta \backslash N_1$, for some $G_\delta$, $m(N_1) = 0$.
			
			\item[(\rmnum{2})]if and only if $E = F_\sigma \backslash N_2$, for some $F_\sigma$, $m(N_2) = 0$.
		\end{enumerate}
	
		\vspace{2em}
		\begin{proof}
			Clearly E is measurable whenever it satisfies either (\rmnum{1}) or (\rmnum{2}).
			\begin{enumerate}
				\item[(\rmnum{1})]Since $E$ is measurable, $\exists$ open sets $O_n \supset E$, $\st$
				\begin{align}
					m(O_n \backslash E) \leq \frac{1}{n}
				\end{align}
				Let $O = \overset{\infty}{\underset{j = 1}{\bigcap}}{O_j}$, then
				\begin{align}
					m(O \backslash E) \leq \frac{1}{n} , \,\, \forall n \in \N
				\end{align}
				Let $n \to \infty$, we get $m(O \backslash E) = 0$. Let $G_\delta = O , \,\, N_1 = O \backslash E$. Then $E = G_\delta \backslash N_1$.
				
				\item[(\rmnum{2})]Similarly, we can easily proof it by Thm\ref{thm 1.3.4}(\rmnum{2}).
			\end{enumerate}
		\end{proof}
	\end{thm}

\newpage
\section{$Non-measurable \,\, sets$}
	在这一节我们将介绍$\R$ 上一个经典的不可测集$Vitali \,\, set$,并说明$\R$ 上每个正测度集都有不可测子集.
\paragraph{Vitali set}
	Let $x , y \in [0 , 1]$. Write $x \sim y$ $\Leftrightarrow$ $x - y \in \Q$.\\
	$\Rightarrow$ 容易验证$\sim$ 为 an equivalence relation.\\
	$\Rightarrow$ $\sim$ partions [0 , 1]. 记[0 , 1]上等价类为$\varepsilon_{\alpha}$,则
	\begin{align}
		[0 , 1] = \bigsqcup_{\alpha}{\varepsilon_{\alpha}} , \,\, \{ \varepsilon_{\alpha} \}_{\alpha} \,\, are \,\, disjoint
	\end{align}
	$\Rightarrow$ By \textcolor{red}{\textbf{the Axiom of Choice}}, we can choose exactly one element $x_{\alpha}$ from each $\varepsilon_{\alpha}$.\\
	$\Rightarrow$ Let $\mathcal{N} = \{ x_{\alpha} \}_{\alpha}$. Then $\mathcal{N}$ is the Vitali set.
	
	\begin{thm}\label{thm 1.5.1}
		$\mathcal{N}$ is not measurable.
		
		\vspace{2em}
		\begin{proof}
			Assume that $\mathcal{N}$ is measurable. Let $\{ r_k \}_{k = 1}^{\infty}$ be an enumeration of $\Q \cap [-1 , 1]$.\\
			Define
			\begin{align}
				\mathcal{N}_k \coloneqq N + r_k = \{ x_\alpha + r_k \}_{\alpha}
			\end{align}
			Then we shall proof that $\{ \mathcal{N}_k \}_{k = 1}^{\infty}$ are disjoint, and $[0 , 1] \subset \overset{\infty}{\underset{k = 1}{\bigcup}}{\mathcal{N}_k} \subset [-1 , 2]$.
			
			\begin{itemize}
				\item If $\mathcal{N}_k \cap \mathcal{N}_m \neq \varnothing$, then $\exists x_\alpha , x_\beta \in \mathcal{N} , \,\, r_k , r_m \in \Q \cap [-1 , 1]$, $\st$
				\begin{align}
					x_\alpha + r_k = x_\beta + r_m
				\end{align}
				Then $x_\alpha - x_\beta = r_m - r_k \in \Q$ $\Rightarrow$ $x_\alpha \sim x_\beta$ $\Rightarrow$ $x_\alpha , x_\beta \in \varepsilon_{\alpha}$ or $x_\alpha , x_\beta \in \varepsilon_\beta$ $\Rightarrow$ $x_\alpha = x_\beta$ and $r_k = r_m$.\\
				Therefore, $\mathcal{N}_k = \mathcal{N}_m$.
				
				\item Since $r_k \in [-1 , 1]$, $\mathcal{N}_k \in [-1 , 2]$, $\forall k$. Therefore,
				\begin{align}
					\bigcup_{k = 1}^{\infty}{\mathcal{N}_k} \subset [-1 , 2]
				\end{align}
			
				\item $\forall x \in [0 , 1]$. Since $\{ \varepsilon_\alpha \}_{\alpha}$ partions $[0 , 1]$, there exists $\alpha_0$, $\st$
				\begin{align}
					x \in \varepsilon_{\alpha_0} , \,\, x \sim x_{\alpha_0}
				\end{align}
				which means $x - x_{\alpha_0} \in \Q \cap [-1 , 1]$. Then $\exists k_0 \in \N$, $\st$
				\begin{align}
					x - x_{\alpha_0} = r_{k_0} \,\, \Rightarrow \,\, x \in \mathcal{N}_{k_0}
				\end{align}
				Therefore, 
				\begin{align}
					[0 , 1] \subset \bigcup_{k = 1}^{\infty}{\mathcal{N}_k}
				\end{align}
			\end{itemize}
			Since $\{ \mathcal{N}_k \}_{k = 1}^{\infty}$ are disjoint, we get
			\begin{align}
				m([0 , 1]) \leq \sum_{k = 1}^{\infty}{m(\mathcal{N}_k)} \leq m([-1 , 2])
			\end{align}
			Since $\mathcal{N}_k$ is a translate of $\mathcal{N}$, we have $m(\mathcal{N}) = m(\mathcal{N}_k)$ for each k. Then
			\begin{align}
				1 \leq \sum_{k = 1}^{\infty}{m(\mathcal{N})} \leq 3 \,\, \Rightarrow \,\, Neither \,\, m(\mathcal{N}) = 0 \,\, nor \,\, m(\mathcal{N}) > 0 \,\, is \,\, possible.
			\end{align}
			Therefore, it's a contradiction. $\mathcal{N}$ is non-measurable.
		\end{proof}
	\end{thm}

\vspace{2em}
\paragraph{正测度集必有不可测子集}
	下面要证明一个结论,即$\R$ 上任一正测度集必有不可测子集. 这实际上为书\footnote{参考书籍:\textbf{《$Real \,\, Analysis -- Measure \,\, Theroy, \,\, Integration, \,\, \& \,\, Hilbert \,\, Spaces$》--- $Elias \,\, M. \,\, Stein$}}Exercises of Chapter 1 的第32题(b).
	
	\begin{proposition}\label{prop 1.5.1}
		Let $\mathcal{N}$ denote the non-measurable subset of $[0 , 1]$ constructed in Thm\ref{thm 1.5.1}.
		\begin{enumerate}
			\item[(a)]If $E$ is a measurable subset of $\mathcal{N}$, then $m(E) = 0$.
			
			\item[(b)]If $G \subset \R$ with $m_{*}(G) > 0$, then there exists a subset of $G$ is non-measurable.
		\end{enumerate}
	
		\vspace{2em}
		\begin{proof}
			\begin{enumerate}
				\item[(a)]Note $\mathcal{N} = \{ x_\alpha \}_{\alpha \in \mathcal{A}}$, then $E = \{ x_\beta \}_{\beta \in \mathcal{B} \subset \mathcal{A}}$. Similarly, we can proof
				\begin{align}
					\bigcup_{k = 1}^{\infty}{E_k} \subset [-1 , 2]
				\end{align}
				Since $\{ E_k \}_{k = 1}^{\infty}$ are disjoint, and $E_k$ is a translate of $E$, we get
				\begin{align}
					\sum_{k = 1}^{\infty}{m(E)} \leq 3 \,\, \Rightarrow \,\, m(E) = 0
				\end{align}
			
				\item[(b)]Let $\Q = \{ r_k \}_{k = 1}^{\infty}$, $\mathcal{N}_k = \mathcal{N} + r_k$, then
				\begin{align}
					\R = \bigsqcup_{k = 1}^{\infty}{\mathcal{N}_K}
				\end{align}
				Suppose $G$ is measurable. Then
				\begin{align}
					G = G \cap \R = \bigsqcup_{k = 1}^{\infty}{(G \cap \mathcal{N}_k)}
				\end{align}
				If $G \cap \mathcal{N}_k$ is measurable, then $G \cap \mathcal{N}_k \subset \mathcal{N}_k$ is a subset of a non-measurable set $\mathcal{N}_k$.\\
				By the previous (a), we get
				\begin{align}
					m(G \cap \mathcal{N}_k) = 0
				\end{align}
				Therefore, there exists $k_0 \in \N$, $\st G \cap \mathcal{N}_{k_0} \subset G$ is a non-measurable subset of $G$.\\
				(otherwise $m(G) = 0$ contradicts)
			\end{enumerate}
		\end{proof}
	\end{proposition}
 	



	%  ############################
	\ifx\allfiles\undefined
\end{document}
\fi