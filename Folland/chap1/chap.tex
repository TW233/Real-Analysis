\ifx\allfiles\undefined
\documentclass[12pt, a4paper,oneside, UTF8]{ctexbook}
\usepackage[dvipsnames]{xcolor}
\usepackage{amsmath}   % 数学公式
\usepackage{amsthm}    % 定理环境
\usepackage{amssymb}   % 更多公式符号
\usepackage{graphicx}  % 插图
%\usepackage{mathrsfs}  % 数学字体
%\usepackage{newtxtext,newtxmath}
%\usepackage{arev}
\usepackage{kmath,kerkis}
\usepackage{newtxtext}
\usepackage{bbm}
\usepackage{enumitem}  % 列表
\usepackage{geometry}  % 页面调整
%\usepackage{unicode-math}
\usepackage[colorlinks,linkcolor=black]{hyperref}

\usepackage{wrapfig}


\usepackage{ulem}	   % 用于更多的下划线格式,
					   % \uline{}下划线,\uuline{}双下划线,\uwave{}下划波浪线,\sout{}中间删除线,\xout{}斜删除线
					   % \dashuline{}下划虚线,\dotuline{}文字底部加点


\graphicspath{ {flg/},{../flg/}, {config/}, {../config/} }  % 配置图形文件检索目录
\linespread{1.5} % 行高

% 页码设置
\geometry{top=25.4mm,bottom=25.4mm,left=20mm,right=20mm,headheight=2.17cm,headsep=4mm,footskip=12mm}

% 设置列表环境的上下间距
\setenumerate[1]{itemsep=5pt,partopsep=0pt,parsep=\parskip,topsep=5pt}
\setitemize[1]{itemsep=5pt,partopsep=0pt,parsep=\parskip,topsep=5pt}
\setdescription{itemsep=5pt,partopsep=0pt,parsep=\parskip,topsep=5pt}

% 定理环境
% ########## 定理环境 start ####################################
\theoremstyle{definition}
\newtheorem{defn}{\indent 定义}[section]

\newtheorem{lemma}{\indent 引理}[section]    % 引理 定理 推论 准则 共用一个编号计数
\newtheorem{thm}[lemma]{\indent 定理}
\newtheorem{corollary}[lemma]{\indent 推论}
\newtheorem{criterion}[lemma]{\indent 准则}

\newtheorem{proposition}{\indent 命题}[section]
\newtheorem{example}{\indent \color{SeaGreen}{例}}[section] % 绿色文字的 例 ,不需要就去除\color{SeaGreen}{}
\newtheorem*{rmk}{\indent \color{red}{注}}

% 两种方式定义中文的 证明 和 解 的环境:
% 缺点:\qedhere 命令将会失效【技术有限,暂时无法解决】
\renewenvironment{proof}{\par\textbf{证明.}\;}{\qed\par}
\newenvironment{solution}{\par{\textbf{解.}}\;}{\qed\par}

% 缺点:\bf 是过时命令,可以用 textb f等替代,但编译会有关于字体的警告,不过不影响使用【技术有限,暂时无法解决】
%\renewcommand{\proofname}{\indent\bf 证明}
%\newenvironment{solution}{\begin{proof}[\indent\bf 解]}{\end{proof}}
% ######### 定理环境 end  #####################################

% ↓↓↓↓↓↓↓↓↓↓↓↓↓↓↓↓↓ 以下是自定义的命令  ↓↓↓↓↓↓↓↓↓↓↓↓↓↓↓↓

% 用于调整表格的高度  使用 \hline\xrowht{25pt}
\newcommand{\xrowht}[2][0]{\addstackgap[.5\dimexpr#2\relax]{\vphantom{#1}}}

% 表格环境内长内容换行
\newcommand{\tabincell}[2]{\begin{tabular}{@{}#1@{}}#2\end{tabular}}

% 使用\linespread{1.5} 之后 cases 环境的行高也会改变,重新定义一个 ca 环境可以自动控制 cases 环境行高
\newenvironment{ca}[1][1]{\linespread{#1} \selectfont \begin{cases}}{\end{cases}}
% 和上面一样
\newenvironment{vx}[1][1]{\linespread{#1} \selectfont \begin{vmatrix}}{\end{vmatrix}}

\def\d{\textup{d}} % 直立体 d 用于微分符号 dx
\def\R{\mathbb{R}} % 实数域
\def\N{\mathbb{N}} % 自然数域
\def\C{\mathbb{C}} % 复数域
\def\Z{\mathbb{Z}} % 整数环
\def\Q{\mathbb{Q}} % 有理数域
\newcommand{\bs}[1]{\boldsymbol{#1}}    % 加粗,常用于向量
\newcommand{\ora}[1]{\overrightarrow{#1}} % 向量

% 数学 平行 符号
\newcommand{\pll}{\kern 0.56em/\kern -0.8em /\kern 0.56em}

% 用于空行\myspace{1} 表示空一行 填 2 表示空两行  
\newcommand{\myspace}[1]{\par\vspace{#1\baselineskip}}

%s.t. 用\st就能打出s.t.
\DeclareMathOperator{\st}{s.t.}

%罗马数字 \rmnum{}是小写罗马数字, \Rmnum{}是大写罗马数字
\makeatletter
\newcommand{\rmnum}[1]{\romannumeral #1}
\newcommand{\Rmnum}[1]{\expandafter@slowromancap\romannumeral #1@}
\makeatother
\begin{document}
	% \title{{\Huge{\textbf{$Real \,\, Analysis$}}}\\
		\Large{\textbf{$Measure \,\, Theory , \,\, Integration , \,\, \& \,\, Hilbert \,\, Spaces$}}\footnote{参考书籍:\\
			\hspace*{4em} \textbf{《$Real \,\, Analysis -- Measure \,\, Theroy, \,\, Integration, \,\, \& \,\, Hilbert \,\, Spaces$》--- $Elias \,\, M. \,\, Stein$} \\
			\hspace*{4em} \textbf{《$Real \,\, Analysis -- Modern \,\, Techniques \,\, and \,\, Their \,\, Applications$》--- $Gerald \,\, B. \,\, Folland$}}}
\author{$-TW-$}
\date{\today}
\maketitle                   % 在单独的标题页上生成一个标题

\thispagestyle{empty}        % 前言页面不使用页码
\begin{center}
	\Huge\textbf{序}
\end{center}


\vspace*{3em}
\begin{center}
	\large{\textbf{天道几何,万品流形先自守;}}\\
	
	\large{\textbf{变分无限,孤心测度有同伦。}}
\end{center}

\vspace*{3em}
\begin{flushright}
	\begin{tabular}{c}
		\today \\ \small{\textbf{长夜伴浪破晓梦,梦晓破浪伴夜长}}
	\end{tabular}
\end{flushright}


\newpage                      % 新的一页
\pagestyle{plain}             % 设置页眉和页脚的排版方式(plain:页眉是空的,页脚只包含一个居中的页码)
\setcounter{page}{1}          % 重新定义页码从第一页开始
\pagenumbering{Roman}         % 使用大写的罗马数字作为页码
\tableofcontents              % 生成目录

\newpage                      % 以下是正文
\pagestyle{plain}
\setcounter{page}{1}          % 使用阿拉伯数字作为页码
\pagenumbering{arabic}
\setcounter{chapter}{0}    % 设置 -1 可作为第零章绪论从第零章开始 
	\else
	\fi
	%  ############################ 正文部分
\chapter{课程要求}
	\begin{itemize}
		\item \textbf{任课教师}:刘小川
		
		\item \textbf{辅导时间}:希腊奶
		
		\item \textbf{办公室}:数学楼$206$
		
		\item \textbf{$Email$}:$liuxiaochuan@mail.xjtu.edu.cn$
		
		\item \textbf{总评成绩组成}:期末$70\%$ + 平时$30\%$
		
		\item 考试英文题,答题中 / 英
	\end{itemize}

\chapter{集合论}

\section{集合列的上下极限}

\subsection{集合族的上下确界}
\paragraph{定义}
	首先,对于任意一族集合$\{ E_n \}_{n \in I}$,我们给出其上界和上确界的定义:
	\begin{defn}\label{def 1.1.1}
		对于$\{ E_n \}_{n \in I}$,若集合$F$ 满足$E_n \subseteq F , \forall n \in I$,则称$F$ 为集合族$\{ E_n \}_{n \in I}$ 的\underline{\textbf{上界}}
	\end{defn}

	\begin{defn}\label{def 1.1.2}
		$\{ E_n \}_{n \in I}$ 的上界的交成为$\{ E_n \}_{n \in I}$ 的\underline{\textbf{上确界}},即
		\begin{align}
			\sup_{n \in I}{E_n} = \bigcup_{n \in I}{E_n}
		\end{align}
	\end{defn}
	类似的可给出下界及下确界的定义.\\
	
\paragraph{性质}
	下面给出两条关于上下确界的显然的性质:
	\begin{proposition}\label{prop 1.1.1}
		若指标集$I_1 \supseteq I_2$,则:
		\begin{align}
			\sup_{n \in I_1}{E_n} &\supseteq \sup_{n \in I_2}{E_n}\\
			\inf_{n \in I_1}{E_n} &\subseteq \inf_{n \in I_2}{E_n}
		\end{align}
	\end{proposition}

\subsection{集合列的上下极限}
\paragraph{定义}
	我们取
	\begin{align}
		I_k \coloneqq \{ n \in \N \mid n \geq k \}
	\end{align}
	则$\{ I_k \}_{k = 1}^{\infty}$ 单调,从而根据命题\ref{prop 1.1.1}可知,集合列$\{ \sup_{n \in I_k}{E_n} \}_{k = 1}^{\infty} , \{ \inf_{n \in I_k}{E_n} \}_{k = 1}^{\infty}$ 也单调(前者递减,后者递增),从而可定义任一集合列的上下极限:
	\begin{defn}\label{def 1.1.3}
		\begin{align}
			\limsup_{n \to \infty}{E_n} 
			= \varlimsup_{n \to \infty}{E_n} 
			&\coloneqq \lim_{k \to \infty}{\sup_{n \in I_k}{E_n}} 
			= \bigcap_{k = 1}^{\infty}{\bigcup_{n = k}^{\infty}{E_n}}\\
			\liminf_{n \to \infty}{E_n} 
			= \varliminf_{n \to \infty}{E_n} 
			&\coloneqq \lim_{k \to \infty}{\inf_{n \in I_k}{E_n}} 
			= \bigcup_{k = 1}^{\infty}{\bigcap_{n = k}^{\infty}{E_n}}
		\end{align}
	\end{defn}

\paragraph{性质}
	下面给出集合列的上下极限的性质,也可视作等价定义 / 不同观点
	\begin{proposition}\label{prop 1.1.2}
		\begin{align}
			\varlimsup_{n \to \infty}{E_n} &= \{ x \mid x \in E_n \text{对无穷多个$n$ 成立} \}\\
			\varliminf_{n \to \infty}{E_n} &= \{ x \mid x \in E_n \text{对除有限个$n$ 成立} \}
		\end{align}
	\end{proposition}

	根据$Demorgan$ 定律可得,
		\[\varliminf_{n \to \infty}{E_n} = (\varlimsup_{n \to \infty}{E_{n}^{c}})^c\]
		
\section{$Descartes$ 积的推广}
\paragraph{引入}
	首先我们回忆两个(有限个)集合的$Descartes$积的定义
	\begin{align}
		A \times B \coloneqq \{ (x , y) \mid x \in A , \,\, y \in B \}
	\end{align}
	此处定义的$Descartes$ 积与普通的集合的一个显著的区别就是他是\textbf{有序的},这里的\textbf{“有序对”}$(x , y)$与$(y , x)$ 并不相同,这就引出了几个问题:
	\begin{itemize}
		\item 什么是$(x , y)$,即“有序对”的定义是什么?
		
		\item $x , \,\, y$ 的顺序是否重要?\\
		或者对于更一般的一族集合的$Descartes$ 积是否仍可定义\textbf{“顺序”}?
	\end{itemize}
	在解答这些问题之前,我们先来引入一个函数
	\begin{align}
		f : \{ 1 , 2 \} &\longrightarrow A \cup B\\
		1 &\longmapsto x \in A\\
		2 &\longmapsto y \in B
	\end{align}
	则此时函数$f$ 已经给出了我们上面所需的\textbf{“序关系”},即可用以表示$(x , y)$\\
	但这时又冒出了几个新的疑惑:
	\begin{itemize}
		\item 指标集$\{ 1 , 2 \}$ 的选取是否重要?
		\begin{rmk}
			此处的回答显然为否,即我们选取指标集时不应牵扯到角标,比如此处可用$\{ 1 , 2 \}$,也可用$\{ 3 , 4 \}$,或是$\{ c , d \}$,即只需指标集中\textbf{“元素的个数”}相同,而无需考虑具体形式
		\end{rmk}
		
		\item 指标集是否必须为\textbf{有限}集?或是\textbf{可数}集?
	\end{itemize}
	
\paragraph{定义}
	为了解答上述疑惑,下面我们给出更一般的$Descartes$积的定义:
	\begin{defn}\label{def 1.2.1}
		设$J$ 为一个指标集,$\{ E_n \}_{n \in J}$ 为一族集合,定义集合$T$
		\begin{align}
			T \coloneqq \left\{ f : I \longrightarrow \bigcup_{n \in J}{E_n} \,\, \Big| \,\, I \approx J \right\}
		\end{align}
		并在集合$T$ 上定义等价关系$\sim$:
		\begin{align}
			f \sim g \Longleftrightarrow \exists \text{双射} \varphi , \,\, \st f \circ \varphi = g
		\end{align}
		在此基础上,定义集合族$\{ E_n \}_{n \in J}$ 的\underline{\textbf{$Descartes$积}}:
		\begin{align}
			\prod_{n \in J}{E_n} \coloneqq \left\{ \overline{f} \,\, \Big| \,\, f : J \longrightarrow \bigcup_{n \in J}{E_n} , \,\, \forall n \in J , \,\, f(n) \in E_n \right\}
		\end{align}
		\begin{rmk}
			\begin{itemize}
				\item $I \approx J$ 表示集合$I$ 与$J$ 等势,即存在$I$ 到$J$ 的双射
				
				\item $\overline{f}$ 表示$f$ 在集合$T$ 上的等价类,注意此处$Descartes$积中的$\overline{f}$ 剔除了$f$ 在$T$ 的等价类中不满足条件“$\forall n \in J , \,\, f(n) \in E_n$”的部分函数
				
				\item 此定义可理解为:\\
				从每个$E_n$ 中各选一者一一置于一些\textbf{不记次序}的空位中,即构成一个\textbf{多重集}
				
				\item 这里$T$ 上的等价关系$\sim$ 保证了$Descartes$积中函数$f$ 指标集的选取只需考虑\textbf{集合的势}相等,即\textbf{元素的个数}相同
			\end{itemize}
		\end{rmk}
	\end{defn}
	
\paragraph{推广}
	事实上,推广后的定义已不包含集合的序概念,此时再将推广后的$Descartes$积与传统意义上在\textbf{可列集(有限 / 可数)}上定义的$Descartes$积进行对比:\\
	设$J$ 是可列的,可先将$J$ 中元素排序为$j_1 , \,\, j_2 , \,\, j_3 , \cdots$,由此回到“传统的”$Descartes$积:
	\begin{align}
		E_{j_1} \times E_{j_2} \times E_{j_3} \times \cdots
	\end{align}
	事实上,该定义即为定义\ref{def 1.2.1}中$\prod_{n \in J}{E_n}$ 的一个代表元集\\
	同时,在此基础上,我们还可将传统的\textbf{二元关系}拓展为\textbf{多元关系(即为$Descartes$ 积的子集)}
	
\newpage

\section{序关系}
\subsection{偏序,全序,预序}
	首先回顾关系(二元关系)的概念:
	\begin{defn}\label{def 1.3.1}
		设$X , \,\, Y$ 是两个集合,如果集合$R$ 是$X$ 与$Y$ 的$Descartes$ 积的子集,即
		\begin{align}
			R \subseteq X \times Y
		\end{align}
		则称$R$ 是从$X$ 到$Y$ 的一个\underline{\textbf{二元关系(一般称作关系)}}.\\
		于是,若$(x , y) \in R$,我们称$x$ 与$y$ 是$R-$ 相关的,记作$xRy$.
	\end{defn}

\paragraph{偏序}
	此时便能给出偏序的定义:
	\begin{defn}
		设$X$ 为一个集合,满足如下三条公理的关系$R \subseteq X \times X$ 称作$X$ 上的一个\underline{\textbf{偏序关系}}:
		\begin{enumerate}
			\item $if \,\, xRy , \,\, yRz \Rightarrow xRz \,\,$(传递性)
			
			\item $if \,\, xRy , \,\, yRx \Rightarrow x = y \,\,$(反对称性)
			
			\item $xRx , \,\, \forall x \in X \,\,$(自反性)
		\end{enumerate}
	\end{defn}

	\begin{example}\label{ex 1.3.1}
		常见的偏序关系有:$\leq , \,\, \geq , \,\, \subseteq , \,\, \supseteq$,通常把一般的偏序关系记作小于等于$\leq$,上述定义是对常见的偏序关系的推广.
	\end{example}

	\begin{rmk}
		\textbf{偏序关系}是由\textbf{等价关系}所衍生出来的,即先有了相等的概念后才能定义偏序关系.每一个等价关系可以衍生出很多偏序关系,实际上由同一个等价关系所衍生出的偏序关系并不是完全独立的,而是成对出现的(类似于$\subseteq$ 与$\supseteq$).
		\begin{example}\label{ex 1.3.2}
			由上述定义的偏序关系$R$ 可得到一个对偶的偏序关系$R^{'}$,其有如下的关系:
			\begin{align}
				xRy \Leftrightarrow yR^{'}x
			\end{align}
		\end{example}
	\end{rmk}

	下面给出一个偏序集的例子
	\begin{example}\label{ex 1.3.3}
		记全体复数构成集合$\C$,则$(\C , \leq)$ 是偏序集
	\end{example}
	\begin{rmk}
		在例\ref{ex 1.3.3}中,我们不能说形如$a + bi(b \neq 0)$ 的元素之间不满足传递性/反对称性,\uwave{因为形如$a + bi(b \neq 0)$ 两个元素之间没有序关系},此处实际只需考虑$\C$ 中实数之间的序关系\\
		(之所以称偏序集中要求部分元素之间存在序关系,是因为除了反身性以外,其前提均要求选取的对象之间存在序关系。)
	\end{rmk}

\newpage
\paragraph{全序}
	在偏序的基础上,可再进一步地给出全序的概念:
	\begin{defn}\label{def 1.3.3}
		设$X$ 为一个集合,$R$ 为$X$ 上的一个偏序关系,如果$R$ 再同时满足以下性质:
		\begin{align}
			\forall x , \,\, y \in X , \,\, \st xRy \,\, or \,\, yRx
		\end{align}
		则称$R$ 为集合$X$ 上的一个\underline{\textbf{全序关系}}
		\begin{rmk}
			通俗地讲,若$X$ 中任意两个元素之间都满足关系$R$,即任意两个元素之间都可比较,则$(X , R)$ 为一个全序集
		\end{rmk}
	\end{defn}
	
	下面给出一个全序集的例子
	\begin{example}\label{ex 1.3.4}
		设集合$P = \{ \varnothing , \{ a \} , \{a , b \} , \{ a , b , c \} \}$,则$(P , \subseteq)$ 构成全序集
	\end{example}

\paragraph{预序}
	\begin{defn}\label{def 1.3.4}
		设$X$ 为一个集合,$R$ 为$X$ 上的一个二元关系,若$R$ 只满足自反性和传递性,即
		\begin{enumerate}
			\item $xRx , \,\, \forall x \in X \,\,$(自反性)
			
			\item $if \,\, xRy , \,\, yRz \Rightarrow xRz \,\,$(传递性)
		\end{enumerate}
		则称$R$ 为集合$X$ 上的一个\underline{\textbf{预序关系}}
		\begin{rmk}
			由定义可知,\textbf{全序集}一定是\textbf{偏序集},\textbf{偏序集}一定是\textbf{预序集}
		\end{rmk}
	\end{defn}

\subsection{极大元/极小元,上界/下界,良序}
\paragraph{极大元/极小元}
	下面给出偏序集上极小元的定义
	\begin{defn}\label{def 1.3.5}
		设$X$ 为一个集合,$\prec$ 为$X$ 上的一个偏序关系,如果存在$x \in X , \,\, \st$
		\begin{align}
			\forall y \in X , \,\, if \,\, y \preceq x , \,\, then \,\, y = x
		\end{align}
		则称$x$ 为$X$ 的一个\underline{\textbf{极小元}}
		\begin{rmk}
			\begin{enumerate}
				\item 极小元即表示集合中小于或等于它的元素只有它本身,以下为一个等价定义:
				\begin{align}
					\not\exists y \in X , \,\, y \neq x , \,\, \st y \prec x
				\end{align}
			
				\item 并不一定$X$ 中所有的元素都可与$x$ 进行比较,即可以有很多元素与$x$ 没有关系(不可比较大小)
				
				\item 对于任一偏序集$(X , \prec)$,极小元的存在性和唯一性都不一定成立
			\end{enumerate}
		\end{rmk}
	\end{defn}
	同理可给出极大元的定义.
	
\paragraph{上界/下界}
	下面给出下界的定义
	\begin{defn}\label{def 1.3.6}
		设$X$ 为一个集合,$\prec$ 为$X$ 上的一个偏序关系,子集$E \subseteq X$,如果存在$x \in X , \,\, \st$
		\begin{align}
			x \preceq y , \,\, \forall y \in E
		\end{align}
		则称$E$ (在$X$ 中)有下界,$x$ 称为$E$ 的一个\underline{\textbf{下界}}
		\begin{rmk}
			集合$E$ 中的每一个元素$y$ 都与下界$x$ 有关系(与极小元的区别)
		\end{rmk}
	\end{defn}
	同理可给出上界的定义.
	
\paragraph{良序}
	在定义了极小元的基础上,可以进一步来给出良序的定义.
	\begin{defn}\label{def 1.3.7}
		设$(X , \prec)$ 为全序集,如果对于$\forall Y \subseteq X , \,\, Y \neq \varnothing$,$Y$ 有极小元,则称$\prec$ 为$X$ 上的一个\underline{\textbf{良序关系}}
	\end{defn}
	
\subsection{保序同构,序型}





\newpage
\section{$Hausdorff$ 极大原理,$Zorn$ 引理}
	\begin{center}
		注意这两个都是\textbf{公理}性质,是无法被证明的,只能互相推导
	\end{center}

\paragraph{$Hausdorff$ 极大原理}
	下面给出$Hausdorff$ 极大原理的叙述.
	\begin{thm}\label{thm 1.4.1}
		任一偏序集都有极大的全序子集.
		\begin{rmk}
			此处的\textbf{“极大”}指的是,对于集合$\{ \text{该偏序集的所有全序子集} \}$,在包含$\subseteq$ 的偏序关系下的极大元
		\end{rmk}
	\end{thm}

\vspace*{2em}
\paragraph{$Zorn$ 引理}
	下面给出$Zorn$ 引理的叙述.
	\begin{thm}\label{thm 1.4.2}
		若偏序集$X$ 的每个全序子集都有上界,则$X$ 有极大元.
		\begin{rmk}
			此处的上界只需满足存在性,而无需满足唯一性.
		\end{rmk}
	\end{thm}

\vspace*{2em}
\paragraph{相互推导}
	事实上,$Hausdorff$ 极大原理和$Zorn$ 引理是等价的.
	\begin{proof}
		\begin{enumerate}
			\item[“$\Rightarrow$”:] 设$(X , \prec)$ 为一个偏序集,\\
			根据$Hausdorff$ 极大原理,在包含关系$\subseteq$ 下,得到极大全序子集$Y$.\\
			根据$Zorn$ 引理的假设,$Y \subseteq X$ 存在上界$x$,则$x \in Y$\\
			(否则$Y \cup \{ x \}$ 构成的全序子集与$Y$ 的极大性矛盾)\\
			从而$x$ 即为$X$ 的极大元.(否则若存在更大的$y$,则同理$Y \cup \{ y \}$与$Y$ 极大性矛盾)
			
			\vspace*{1em}
			
			\item[“$\Leftarrow$”:] 设$(X , \prec)$ 为一个偏序集,下面证明$X$ 有极大的全序子集:\\
			记集合$Z$
			\begin{align}
				Z = \{ X \text{的所有全序子集} \}
			\end{align}
			从而集合$Z$ 与包含关系$\subseteq$ 构成了一个偏序集.令
			\begin{align}
				A = \bigcup_{U \in Z}{U}
			\end{align}
			从而$A \subseteq X$ 即为$Z$ 中所有元素的上界.\\
			根据$Zorn$ 引理,$Z$ 在偏序关系$\subseteq$ 下有极大元,也就说明了$X$ 有极大的偏序子集.
			
		\end{enumerate}
	\end{proof}


\newpage
\section{良序原理,选择公理}
	\begin{center}
		对这两个\textbf{公理}的证明需要首先承认$Hausdorff$ 极大原理 / $Zorn$ 引理
	\end{center}
\subsection{良序原理}
	下面给出良序原理的叙述.
	\begin{thm}\label{thm 1.5.1}
		任一非空集必为良序集.(任一非空集存在良序)
	\end{thm}
	\begin{proof}
		设$X$ 是个非空集,考虑$X$ 的所有子集的良序构成的集合$W$.\\
		注意到$W$ 中的每个元素,即为各良序关系$\prec_1 , \,\, \prec_2$,都附着着其对应的$X$ 的子集$E_1 , \,\, E_2$\\
		(因为对于不同的子集$E_1 , \,\, E_2$,即使在相同的位置其良序关系相同(即$x_1 \prec_1 x_2 \,\, \&\& \,\, x_1 \prec_2 x_2$),但其在整个子集上的良序关系还是不同的)\\
		因此$W$ 中的元素应当表述为各良序关系(主体)与其对应的子集构成的有序对$(\prec , E)$\\
		在$W$ 中引入这样的偏序关系,记作$\leq$:
		\begin{align}
			(\prec_1 , E_1) \leq (\prec_2 , E_2) \Leftrightarrow 
			\begin{cases}
				E_1\subseteq E_2\\
				\prec_2\mid_{E_1}^{}=\prec_1\\
				\forall x\in E_2 \backslash E_1, \,\, y\in E_1, \,\, y \prec_2 x\\
			\end{cases}
		\end{align}
		也就是说,$\prec_2$ 是$\prec_1$ 的延拓,$\prec_1$ 是$\prec_2$ 在$E_1$ 上的限制,同时$E_2$ 超出$E_1$ 的部分在$\prec_2$ 的意义下总是比$E_1$ 中的元素更大.\\
		下面我们尝试运用$Zorn$ 引理(定理\ref{thm 1.4.2})来证明.任取$W$ 的全序子集$Y$,记
		\begin{align}
			Y = \{ \prec_{\alpha} \}_{\alpha \in I}
		\end{align}
		令
		\begin{align}
			E_Y = \bigcup_{\alpha \in I}{E_\alpha}
		\end{align}
		同理可得到该$X$ 的子集$E_Y$ 下的良序关系$\prec_Y$,此$\prec_Y$ 即为$W$ 的全序子集$Y$ 的上界.\\
		根据$Zorn$ 引理(定理\ref{thm 1.4.2}),$W$ 中有极大元$(\prec , E)$.\\
		事实上,此处$E = X$,$\prec$ 即为$X$ 上的一个良序关系.\\
		(反证法.假设$E \neq X$,设$x \in X \backslash E$,此时可定义$x \prec y , \,\, \forall y \in E$,则$E \cup \{ x \}$ 即可得到$X$ 的一个全序子集,$E \cup \{ x \} \in W$,这与$E$ 的极大性矛盾.)
	\end{proof}

\newpage
\subsection{选择公理}
	下面给出选择公理的叙述.
	\begin{thm}\label{thm 1.5.2}
		非空的非空集族的$Descartes$ 积非空.
		\begin{rmk}
			\begin{itemize}
				\item \textbf{“非空的非空集族”}就是指有一族非空集,其中这一族非空集的个数至少为1
				
				\item \textbf{“$Descartes$ 积非空”}大致上说的是可以不计次序从每个非空集中取出一个元素,构成一个多重集(具体可见定义\ref{def 1.2.1})
			\end{itemize}
		\end{rmk}
	\end{thm}

\vspace*{1em}
	下面我们利用良序原理来对选择公理进行证明.
	\begin{proof}
		设$\{ X_\alpha \}_{\alpha \in I}$ 为一族非空集,其中$X_\alpha \neq \varnothing , \,\, I \neq \varnothing$.\\
		根据良序原理(定理\ref{thm 1.5.1}),集合$\underset{\alpha \in I}{\bigcup}{X_\alpha}$ 存在一个良序关系$\prec$\\
		由于$(\underset{\alpha \in I}{\bigcup}{X_\alpha} , \prec)$ 为良序集,因此其非空子集$X_\alpha \subseteq \underset{\alpha \in I}{\bigcup}{X_\alpha}$均有极小元.
		定义映射$f$
		\begin{align}
			f : I &\longrightarrow \bigcup_{\alpha \in I}{X_\alpha}\\
			\alpha &\longmapsto \min_{\prec}{X_\alpha}
		\end{align}
		从而
		\begin{align}
			\overline{f} \in \prod_{\alpha \in I}{X_\alpha} , \,\, \prod_{\alpha \in I}{X_\alpha} \neq \varnothing
		\end{align}
	\end{proof}

\newpage
\section{集合的势$Cardinality$}
\paragraph{引入}
	为了更好地理解势的概念,我们先给出势的比较关系.对于非空集$X , \,\, Y$,我们定义.
	\begin{align}
		\begin{cases}
			card\left( X \right) \leqslant card\left( Y \right)\\
			card\left( X \right) =card\left( Y \right)\\
			card\left( X \right) \geqslant card\left( Y \right)\\
		\end{cases}
	\end{align}
	分别表示存在从$X$ 到$Y$ 的\textbf{单射、双射、满射}.这与常规下集合元素个数的比较是吻合的.
	
\vspace*{2em}
\paragraph{定义}
	此时再去赋予势$card$ 的意义.
	\begin{defn}\label{def 1.6.1}
		设$X$ 为一个集合,定义$X$的\underline{\textbf{势($Cardinality$)}}.
		\begin{align}
			card(X) \coloneqq \{ Y \text{为集合} \mid \text{存在由$X$ 到$Y$ 的单射} \}
		\end{align}
		记$S$ 为全体集合构成的真类,$S^*$ 为全体非空集合构成的真类.在$S^*$ 上定义等价关系$R$:
		\begin{align}
			xRy \Leftrightarrow \text{存在由$X$ 到$Y$ 的双射}
		\end{align}
		则势的概念自然即为$X$ 在关系$R$ 下的等价类,即
		\begin{align}
			card(X) = \overline{X}
		\end{align}
		\begin{rmk}
			规定$card(\varnothing) < card(X) , \,\, card(X) > card(\varnothing) , \,\, \forall X \neq \varnothing$,进而定义中的$S^*$ 可修正为$S$.其中$card(\varnothing) = \{ \varnothing \}$.
		\end{rmk}
	\end{defn}

\vspace*{2em}
\paragraph{严格证明}
	在大致给出了集合的势的概念后,下面对其中的一些概念进行严格的定义和证明.\\
	对于最开始在全集合类$S$ 中引入的关系$\leq$,下面证明其为$S$ 上的一个偏序关系.\\
	事实上,我们还会证明$\leq$ 是$S$ 上的全序关系,即任意两个集合的势都可比较.
	
	\vspace*{1em}
	对偏序关系的三条公理进行一一验证.即\textbf{传递性、自反性、反对称性}.同时证明\textbf{$\leq$ 与$\geq$ 互为逆关系},$\leq$ 同时为\textbf{全序关系}.
	
	\subparagraph{自反性、传递性}
		事实上自反性和传递性的证明是显然的.
		
	\vspace*{2em}
	\subparagraph{逆关系}
		下面的引理证明了$\leq$ 与$\geq$ 互为逆关系.
		\begin{lemma}\label{lemma 1.6.1}
			设$X , \,\, Y \in S^*$,则
			\begin{align}
				card(X) \leq card(Y) \Leftrightarrow card(Y) \geq card(X)
			\end{align}
			\begin{proof}
				即证:存在$X$ 到$Y$ 的单射$f \,\, \Leftrightarrow \,\,$ 存在$Y$ 到$X$ 的满射$g$.
				\begin{enumerate}
					\item[$\Rightarrow$:]由于$f$ 为单设,因此$\forall y \in f(X) , \,\, \exists$ 唯一的$x \in X , \,\, \st y = f(x)$.\\
					于是可构造
					\begin{align}
						g : Y &\longrightarrow X\\
						y &\longmapsto \begin{cases}
							x , \,\, y=f\left( x \right) \hspace*{3em} &y\in f\left( X \right)\\
							x_0 , \,\, \forall x_0\in X                &y\notin f\left( X \right)
						\end{cases}
					\end{align}
					从而$g$ 即为$Y$ 到$X$ 的满射.
					
					\item[$\Leftarrow$:]由于$g$ 为$Y$ 到$X$ 的满射,因此对于$\forall x \in X , \,\, g^{-1}(x) \subseteq Y$ 为$Y$ 的一个非空子集.\\
					记$Y$ 的子集族$Z$为
					\begin{align}
						Z = \{ g^{-1}(x) \mid x \in X \} = \{ g^{-1}(x) \}_{x \in X}
					\end{align}
					由于$X , \,\, Y \in S^*$ 非空,因此$Z$ 为非空的非空集族.\\
					根据选择公理(定理\ref{thm 1.5.2}),$Z$ 的$Descartes$ 积非空,\\
					即存在一个由指标集$X$ 到$\underset{x \in X}{\bigcup}{g^{-1}(x)}$ 的单射$f$
					\begin{align}
						f : X &\longrightarrow \bigcup_{x \in X}{g^{-1}(x)} \subseteq Y\\
						x &\longmapsto f(x)
					\end{align}
					此选择映射$f$ 即为所求单射.
				\end{enumerate}
			\end{proof}
		\end{lemma}
	
		\vspace*{2em}
		\subparagraph{全序关系}
		下面的引理说明了$\leq$ 实际上还是$S$ 上的一个全序关系,其证明具有一定技巧性.
		\begin{lemma}\label{lemma 1.6.2}
			$\forall X , \,\, Y \in S$
			\begin{align}
				card(X) \leq card(Y) \,\, \text{或} \,\, card(Y) \leq card(X)
			\end{align}
			\begin{proof}
				不妨设$X , \,\, Y \in S^*$.(映射实际上为特殊的二元关系).令
				\begin{align}
					\mathcal{I} = \{ f : X_0 \longrightarrow Y \mid X_0 \subseteq X , \,\, f \text{为单射} \} \subseteq X \times Y
				\end{align}
				类比良序原理(定理\ref{thm 1.5.1})的证明,在集合$\mathcal{I}$ 上定义偏序$\subseteq$.
				\begin{align}
					f \subseteq g \Leftrightarrow \begin{cases}
						X_f \subseteq X_g\\
						g \, \Big|_{X_f} = f
					\end{cases}
				\end{align}
				从而对于$\mathcal{I}$ 的每个全序子集$E$,取$X_E = \underset{f \in E}{\bigcup}{X_f}$,其对应的映射$f_E$ 即为$E$ 的上界.\\
				于是偏序集$(\mathcal{I} , \subseteq)$ 满足$Zorn$ 引理(定理\ref{thm 1.4.2})的条件,存在极大元$f : X_0 \longrightarrow Y$.\\
				假设$X_0 \neq X$ 且$f(X_0) \neq Y$,则$\exists x \in X \backslash X_0 , \,\, y \in Y \backslash f(X)$,此时令
				\begin{align}
					f^{'} \mid_{X_0} &= f\\
					f^{'} : x &\longmapsto y
				\end{align}
				从而得到单射$f^{'} \in I$,且$f \subseteq f^{'}$,这与$f$ 的极大性矛盾.\\
				综上,$X_0 = X$ 或$f(X_0) = Y$,即必定存在$X$ 到$Y$的单射或满射.
			\end{proof}
		\end{lemma}
	
	\vspace*{2em}
	\subparagraph{反对称性($Schr\ddot{o}der-Bernstein$ 定理)}
		下面的定理说明了$\leq$ 具有反对称性.其证明技巧性比较强.
		\begin{thm}\label{thm 1.6.3}
			($Schr\ddot{o}der-Bernstein$ 定理)\\
			设$X , \,\, Y \in S$,若$card(X) \leq card(Y)$ 且$card(Y) \leq card(X)$,则
			\begin{align}
				card(X) = card(Y)
			\end{align}
			\begin{proof}
				即已知存在单射$f : X \longrightarrow Y , \,\, g : Y \longrightarrow X$,证明$X$ 与$Y$ 之间存在双射:\\
				考虑$X , \,\, Y$ 的如下划分:\\
				$\forall x \in X$,构造序列
				\begin{align}
					\{ x_n \}_{n = 1}^{\infty} = \{ x , \,\, g^{-1}(x) , \,\, (f^{-1} \circ g^{-1})(x) , \,\, (g^{-1} \circ f^{-1} \circ g^{-1})(x) , \cdots \}
				\end{align}
				则称
				\begin{align}
					\begin{cases}
						x\in X_{\infty}:x_n\neq \varnothing ,\forall n\in \N\\
						x\in X_X:x_{n_0}\in X\\
						x\in X_Y:x_{n_0}\in Y\\
					\end{cases}, \,\, n_0 \coloneqq \max_{x_n \neq \varnothing}{n}
				\end{align}
				类似的,也有$Y_{\infty} , \,\, Y_X , \,\, Y_Y$.容易证明
				\begin{align}
					f(X_\infty) = Y_\infty , \,\, f(X_X) = Y_X , \,\, f(X_Y) = Y_Y
				\end{align}
				从而$X , \,\, Y$ 的三个部分分别可以建立双射,最终$X , \,\, Y$ 之间存在双射.
			\end{proof}
		\end{thm}
	
\newpage
\section{幂集的势,可数}
\subsection{幂集的势}
	通过比较任一集合与其幂集的势,可以得到全集合类$S$ 上的势关系$\leq$ 不存在极大元,即\textbf{不存在某个集合的势最大}.
	\begin{proposition}\label{prop 1.7.1}
		$\forall X \in S^* ,$
		\begin{align}
			card(X) < card(2^X)
		\end{align}
		\begin{proof}
			\begin{itemize}
				\item 首先,$card(X) \leq card(2^X)$.存在$X$ 到$2^X$ 的映射$f$,
				\begin{align}
					f : X &\longrightarrow 2^X\\
					x &\longmapsto \{ x \}
				\end{align}
				从而$f$ 为单射,$card(X) \leq card(2^X)$.
				
				\item 其次,不存在$X$ 到$2^X$ 的满射.$\forall g : X \longrightarrow 2^X$,令
				\begin{align}
					Y \coloneqq \{ x \in X \mid x \notin g(x) \}
				\end{align}
				下面证明:$\not\exists y \in X , \,\, \st g(y) = Y \in 2^X$.\\
				反证法.假设$\exists x_0 \in X , \,\, \st g(x_0) = Y$,则
				\begin{enumerate}
					\item[\rmnum{1}.] 若$x_0 \in Y$,则根据$Y$ 的定义,$x_0 \notin g(x_0) = Y$,这与$x_0 \in Y$ 矛盾.
					
					\item[\rmnum{2}.] 若$x_0 \notin Y$,则$x_0 \notin g(x_0) = Y$,根据$Y$ 的定义,$x_0 \in Y$,矛盾.
				\end{enumerate}
				综上,不存在$X$ 到$2^X$ 的满射.
			\end{itemize}
			$Therefore ,$
			\begin{align}
				card(X) < card(2^X)
			\end{align}
		\end{proof}
	\end{proposition}

\newpage
\subsection{可数}
\paragraph{定义}
	\begin{defn}\label{def 1.7.1}
		设$X$ 为一个集合,则称
		\begin{align}
			X \,\, \text{可数} \Leftrightarrow card(X) \leq card(\N)
		\end{align}
		\begin{rmk}
			常将上述定义的集合称为\textbf{至多可数},即包含\textbf{有限}和\textbf{无限可数}两种情况.
		\end{rmk}
	\end{defn}

\vspace*{2em}
\paragraph{性质}
	下面是可数集的两条重要的性质.
	\begin{proposition}\label{prop 1.7.2}
		\begin{enumerate}
			\item[(\rmnum{1})]\textbf{(可数个可数集的并集是可数集)}\\
			若$\{ X_\alpha \}_{\alpha \in I}$ 满足$\begin{cases}
				I \,\, \text{可数}\\
				\forall \alpha \in I , \,\, X_{\alpha} \,\, \text{可数}
			\end{cases}$,则$\underset{\alpha \in I}{\bigcup}{X_\alpha}$ 可数.
			
			\vspace*{1em}
			
			\item[(\rmnum{2})]\textbf{(无限可数集与自然数集$\N$ 等势)}\\
			若$X$ 可数且$X$ 为无限集,则$card(X) = card(\N)$.
		\end{enumerate}
	\end{proposition}
	
	\begin{example}\label{ex 1.7.1}
		$\Z , \,\, \Q$ 是可数集.
	\end{example}

\newpage
\section{可数集的幂集,连续统}
\paragraph{定义}
	下面给出连续统的定义.
	\begin{defn}\label{def 1.8.1}
		设$X \in S$,则
		\begin{align}
			X \textbf{为\underline{\textbf{连续统}}} \,\, \Leftrightarrow \,\, card(X) = card(\R) \coloneqq c
		\end{align}
	\end{defn}
	\begin{rmk}
		提及连续统,就不得不谈到\textbf{连续统假设($Continuum \,\, Hypothesis$,简记$CH$)}.
		\begin{thm}[\textbf{连续统假设$CH$}]\label{thm 1.8.1}
			$\not\exists X \subseteq \R , \,\, \st$
			\begin{align}
				card(\N) < card(X) < card(\R)
			\end{align}
		\end{thm}
		而对于康托尔提出的这样一个假设,美国数学家科恩在1963年证明:
		\begin{center}
			\textbf{在$ZFC$ 公理系统上,$CH$ 既不可被证明,也不可被证伪}.
		\end{center}
	\end{rmk}

\vspace*{2em}
\paragraph{连续统的势}
	下面给出有关连续统的一个重要的命题,它刻画了连续统与可数集的幂集之间的关系.
	\begin{proposition}\label{prop 1.8.1}
		\begin{align}
			card(2^{\N}) = card(\R) = c
		\end{align}
		\begin{proof}
			具体证明过程见视频\href{https://www.bilibili.com/video/BV1RX4y1R7Yc}{可数集的幂集与连续统}.
		\end{proof}
	\end{proposition}
	由此可得到推论.
	\begin{corollary}\label{cor 1.8.2}
		设$X \in S$,若$card(X) \geq c$,则$X$ 不可数.
	\end{corollary}

\vspace*{2em}
\paragraph{性质}
\begin{proposition}\label{prop 1.8.2}
	若$\{ X_\alpha \}_{\alpha \in I}$ 满足
	\begin{align}
		\begin{cases}
			card\left( I \right) =card\left( \R \right) \,\, (\leq)\\
			\forall \alpha \in I , \,\, card\left( X_{\alpha} \right) = card\left( \R \right) \,\, (\leq)
		\end{cases}
	\end{align}
	则
	\begin{align}
		card\left(\bigcup_{\alpha \in I}{X_\alpha}\right) = card(\R) \,\, (\leq)
	\end{align}
\end{proposition}

	在进行证明之前,先证明以下引理.
	\begin{lemma}\label{lemma 1.8.3}
		若$card(X) \leq card(\R) , \,\, card(Y) \leq card(\R)$,则$card(X \times Y) \leq card(\R)$
	\end{lemma}
	\begin{proof}
		由于$card(X \times Y) \leq card(X \times \R) \leq card(\R \times \R)$,因此\\
		即证$card(\R \times \R) = card(\R)$.由于可列集的幂集与连续统等势(命题\ref{prop 1.8.1}),因此\\
		即证$card(2^{\N} \times 2^{\N}) = card(2^{\N})$.下面分别构造$2^{\N} \times 2^{\N}$ 到$2^{\N}$ 的单射和满射:
		\begin{itemize}
			\item 令
			\begin{align}
				f : 2^{\N} \times 2^{\N} &\longrightarrow 2^{\N} \\
				A \times B &\longmapsto C
			\end{align}
			其中
			\begin{align}
				C \coloneqq \left\{ c \,\, \Big| \,\, c = 
				\begin{cases}
					2a+1,&\forall a\in A\\
					2b,&\forall b\in B
				\end{cases} \right\}
			\end{align}
			从而得到了单射$f : 2^{\N} \times 2^{\N} \longrightarrow 2^{\N}$.
			
			\item 令
			\begin{align}
				g : 2^{\N} \times 2^{\N} &\longrightarrow 2^{\N} \\
				A \times B &\longmapsto C
			\end{align}
			其中
			\begin{align}
				A &\coloneqq \{ a \mid a = \frac{c + 1}{2} , \,\, \forall c \in C \,\, \text{为奇数} \} \\
				B &\coloneqq \{ b \mid b = \frac{c}{2} , \,\, \forall c \in C \,\, \text{为偶数} \}
			\end{align}
			容易证明,$g : 2^{\N} \times 2^{\N} \longrightarrow 2^{\N}$ 为满射,从而得证.
		\end{itemize}
	\end{proof}

	下面对命题\ref{prop 1.8.2}进行证明.
	\begin{proof}
		由于$\forall \alpha \in I , \,\, card\left( X_{\alpha} \right) \leq card\left( \R \right)$,因此存在满射$f_\alpha : \R \longrightarrow X_\alpha$.\\
		令
		\begin{align}
			f : I \times \R &\longrightarrow \bigcup_{\alpha \in I}{X_\alpha} \\
			(\alpha , r) &\longmapsto f_{\alpha}(r)
		\end{align}
		由于$f_\alpha$ 为满射,因此$f : I \times \R \longrightarrow \underset{\alpha \in I}{\bigcup}{X_\alpha}$ 为满射,从而根据引理\ref{lemma 1.8.3}
		\begin{align}
			card(\bigcup_{\alpha \in I}{X_\alpha}) \leq card(I \times \R) \leq card(\R)
		\end{align}
	\end{proof}
	
\newpage
\section{理想实数系及上面的求和}
\subsection{实数系的推广}
	\begin{defn}\label{def 1.9.1}
		\underline{\textbf{理想实数系$\overline{\R}$}} 是对实数系的推广.
		\begin{align}
			\overline{\R} \coloneqq \R \cup \{ -\infty , +\infty \}
		\end{align}
		其中我们规定实数系$\R$ 中,
		\begin{align}
			-\infty < x < +\infty , \,\, \forall x \in \R
		\end{align}
		这样我们便可自然地将$\R$ 上的偏序关系延拓到$\overline{\R}$ 中.
	\end{defn}

\subsection{$\overline{\R}_{\geq 0}$ 中的和}
\paragraph{引入}
	在实数系$\R$ 中,我们所定义的级数求和$\sum$ 都是建立在\textbf{至多可数项}的基础之上,并且它对于\textbf{求和顺序}大多时候是有关系的.\\
	而对于任意一族数,我们就会面临以下的问题:
	\begin{itemize}
		\item 这族数能否求和?
		
		\item 若能求和,则其结果是否与求和顺序有关?
	\end{itemize}

\vspace*{2em}
\paragraph{定义}
	为了解决上述问题,我们在对实数系$\R$ 进行延拓后,在理想实数系$\overline{\R}$ 上对求和$\sum$ 进行推广.\\
	首先考虑非负理想实数系$\overline{\R}_{\geq 0}$ 中的和.(排除了\textbf{求和顺序}的考虑)
	\begin{defn}\label{def 1.9.2}
		$\forall I \in S^* , \,\, f : I \longrightarrow \overline{\R}_{\geq 0}$.定义$f$ 的和$\sideset{}{_f}{\sum}$ 是这样一个映射.
		\begin{align}
			\sideset{}{_f}{\sum} : \mathcal{P}(I) &\longrightarrow \overline{\R}_{\geq 0} \\
			X &\longmapsto \sum_{x \in X}{f(x)} \coloneqq 
			\sup_{F \subseteq X , \,\, F \text{有限}}{\left( \sum_{x \in F}{f(x)} \right)}
		\end{align}
		\begin{rmk}
			\begin{enumerate}
				\item 当$I$ 为可数集时,映射$f : I \longrightarrow \overline{\R}_{\geq 0}$ 可视作非负数列,而此处定义的$\underset{x \in X}{\sum}{f(x)}$ 与正项级数的定义吻合.
				
				\item 此处对于任意一族数的和$\underset{x \in X}{\sum}{f(x)}$ 的定义事实上与$Lesbesgue$ 积分的定义是吻和的.
			\end{enumerate}
		\end{rmk}
	\end{defn}

\newpage
\paragraph{性质}
	下面给出$\overline{\R}_{\geq 0}$ 中的和的几条重要的性质.
	\begin{proposition}\label{prop 1.9.1}
		$Let \,\, X \in S^* , \,\, \forall f : X \longrightarrow \overline{\R}_{\geq 0}$,
		若集合$A \coloneqq \{ x \mid f(x) > 0 \}$ 为不可数集,则
		\begin{align}
			\sum_{x \in X}{f(x)} = +\infty
		\end{align}
		
		\vspace*{2em}
		\begin{proof}
			反证法.假设$\underset{x \in X}{\sum}{f(x)} = \underset{x \in A}{\sum}{f(x)} = M < +\infty$,记
			\begin{align}
				A_n \coloneqq \{ x \in X \mid f(x) \in (\frac{1}{n + 1} , \frac{1}{n}) \}
			\end{align}
			由于$\underset{x \in A}{\sum}{f(x)} = M$ 有界,因此
			\begin{align}
				A \backslash \bigcup_{n = 1}^{+\infty}{A_n} = \{ x \in X \mid f(x) \geq 1 \} &\preceq \N \\
				A_n &\preceq \N 
			\end{align}
			从而
			\begin{align}
				\bigcup_{n = 1}^{+\infty}{A_n} &\preceq \N \\
				A = \left( A \backslash \bigcup_{n = 1}^{+\infty}{A_n} \right) \cup 
				\left( \bigcup_{n = 1}^{+\infty}{A_n} \right) &\preceq \N
			\end{align}
			而这与$A$ 不可数矛盾.
		\end{proof}
	\end{proposition}

	\vspace*{3em}
	下面的这个命题是对\textbf{正项级数的可交换性}在理想实数系$\overline{\R}_{\geq 0}$ 上的推广形式.
	\begin{proposition}\label{prop 1.9.2}
		$Let \,\, X \in S^* , \,\, \forall f : X \longrightarrow \overline{\R}_{\geq 0}$,
		若集合$A \coloneqq \{ x \mid f(x) > 0 \}$ 无穷可数,则
		\begin{align}
			\sum_{x \in X}{f(x)} = \sum_{n = 0}^{+\infty}{f \circ g(n)}
		\end{align}
		其中$g : \N \longrightarrow A$ 为双射.
		
		\vspace*{2em}
		\begin{proof}
			易证.
		\end{proof}
	\end{proposition}
	%  ############################
	\ifx\allfiles\undefined
\end{document}
\fi