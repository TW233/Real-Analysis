\ifx\allfiles\undefined
\input{../config/config}
\begin{document}
	% \input{../config/cover} 
	\else
	\fi
	%  ############################ 正文部分
\chapter{$Measures$}

\newpage
\setcounter{section}{3}
\section{$Outer \,\, Measures$}
	\begin{proposition}\label{prop 2.4.1}
		Let $\mathcal{E} \subset \mathcal{P}(X)$ and $\rho : \mathcal{E} \rightarrow [0 , \infty]$ be such that $\varnothing \in \mathcal{E}$, $X \in \mathcal{E}$ and $\rho(\varnothing) = 0$. For any $A \subset X$, define
		\begin{align}
			\mu^{*}(A) = \inf{\left\{ \sum_{j = 1}^{\infty}{\rho(E_j)} \mid E_j \in \mathcal{E} \,\, and \,\, A \subset \bigcup_{j = 1}^{\infty}{E_j} \right\}}
		\end{align}
		Then $\mu^{*}$ is an outer measure.
	\end{proposition}
	
	\vspace*{10em}
	
	\begin{thm}\label{thm 2.4.1}
		\textbf{Carath\'{e}odory's Theorem}. \\
		If $\mu^{*}$ is an outer measure on $X$, the collection $\mathcal{M}$ of $\mu^{*}$-measurable sets is a $\sigma$-algebra, and the restriction of $\mu^{*}$ to $\mathcal{M}$ is a complete measure.
	\end{thm}
	
	\vspace*{10em}
	
	\begin{proposition}\label{prop 2.4.2}
		If $\mu_0$ is a premeasure on $\mathcal{A}$ and $\mu^{*}$ is an outer measure defined by
		\begin{align}
			\mu^{*}(E) = \inf{\left\{ \sum_{j = 1}^{\infty}{\rho(A_j)} \mid A_j \in \mathcal{A} \,\, and \,\, E \subset \bigcup_{j = 1}^{\infty}{A_j} \right\}}
		\end{align}
		then
		\begin{enumerate}
			\item[a.] $\mu^{*} \mid_\mathcal{A} = \mu_0$;
			
			\item[b.] Every set in $\mathcal{A}$ is $\mu^{*}$-measurable.
		\end{enumerate}
	\end{proposition}
	
	\newpage
	
	\begin{thm}\label{thm 2.4.2}
		Let $\mathcal{A} \subset \mathcal{P}(X)$ be an algebra, $\mu_0$ a premeasure on $\mathcal{A}$, and $\mathcal{M}$ the $\sigma$-algebra generated by $\mathcal{A}$. There exists a measure $\mu$ on $\mathcal{M}$ whose restriction to $\mathcal{A}$ is $\mu_0$ -- namely, $\mu = \mu^{*} \mid_{\mathcal{M}}$ where $\mu^{*}$ is given by 
		\begin{align}
			\mu^{*}(E) = \inf{\left\{ \sum_{j = 1}^{\infty}{\rho(A_j)} \mid A_j \in \mathcal{A} \,\, and \,\, E \subset \bigcup_{j = 1}^{\infty}{A_j} \right\}}
		\end{align}
		If $\nu$ is another measure on $\mathcal{M}$ that extends $\mu_0$, then
		\begin{align}
			\nu(E) \leq \mu(E) , \,\, \forall E \in \mathcal{M}
		\end{align}
		with equality when $\mu(E) < \infty$.\\
		If $\mu_0$ is $\sigma$-finite, then $\mu$ is the unique extension of $\mu_0$ to a measure on $\mathcal{M}$.
	\end{thm}

\newpage

\section{$Borel \,\, Measures \,\, on \,\, the \,\, Real \,\, Line$}
	\begin{proposition}\label{prop 2.5.1}
		Let $F : \R \rightarrow \R$ be increasing and right continuous. If $(a_j , b_j] , j = 1 \sim n$ are disjoint h-intervals, let
		\begin{align}
			\mu_0(\bigcup_{j = 1}^{n}{(a_j , b_j]}) = \sum_{j = 1}^{n}{\left[ F(b_j) - F(a_j) \right]}
		\end{align}
		and let $\mu_0(\varnothing) = 0$. Then $\mu_0$ is a premeasure on the algebra $\mathcal{A}$, where
		\begin{align}
			\mathcal{A} = \left\{ finite \,\, disjoint \,\, unions \,\, of \,\, h-intervals \right\}
		\end{align}
	\end{proposition}
	
	\vspace*{16em}
	
	\begin{thm}\label{thm 2.5.1}
		If $F : \R \rightarrow \R$ is any increasing, right continuous function, there is a unique Borel measure $\mu_F$ on $\R$ such that 
		\begin{align}
			\mu_F((a , b]) = F(b) - F(a) , \,\, \forall a , b
		\end{align}
		If $G$ is another such function, we have
		\begin{center}
			$\mu_F = \mu_G \,\, \Leftrightarrow \,\, F - G$ is constant
		\end{center}
		Conversely, if $\mu$ is a Borel measure on $\R$ that is finite on all bounded Borel sets and we define
		\begin{align}
			F(x) = 
			\begin{cases}
				\mu((0 , x]) , \,\, if \,\, x > 0 \\
				0 , \,\, if \,\, x = 0 \\
				-\mu((x , 0]) , \,\, if \,\, x < 0
			\end{cases}
		\end{align}
		then $F$ is increasing and right continuous, and $\mu = \mu_F$.
	\end{thm}

	\newpage
	
	\begin{lemma}\label{lemma 2.5.2}
		Fix a complete Lebesgue-Stieltjes measure $\mu$ on $\R$ associated to the increasing, right continuous function $F$, and we denote by $\mathcal{M}_\mu$ the domain of $\mu$ ($\mathcal{M}_\mu$ is always strictly larger than $\mathcal{B}_\R$). Then for any $E \in \mathcal{M}_\mu$,
		\begin{align}
			\mu(E) 
			&= \inf{\left\{ \sum_{j = 1}^{\infty}{\left[ F(b_j) - F(a_j) \right]} \mid E \subset \bigcup_{j = 1}^{\infty}{(a_j , b_j]} \right\}} \\
			&= \inf{\left\{ \sum_{j = 1}^{\infty}{\mu((a_j , b_j])} \mid E \subset \bigcup_{j = 1}^{\infty}{(a_j , b_j]} \right\}} \\
			&= \inf{\left\{ \sum_{j = 1}^{\infty}{\mu((a_j , b_j))} \mid E \subset \bigcup_{j = 1}^{\infty}{(a_j , b_j)} \right\}}
		\end{align}
	\end{lemma}
	
	\vspace*{6em}
	
	\begin{thm}\label{thm 2.5.3}
		If $E \in \mathcal{M}_\mu$, then
		\begin{align}
			\mu(E) 
			&= \inf{\left\{ \mu(U) \mid E \subset U \,\, and \,\, U \,\, is \,\, open \right\}} \\
			&= \sup{\left\{ \mu(K) \mid K \subset E \,\, and \,\, K \,\, is \,\, compact \right\}}
		\end{align}
	\end{thm}
	
	\vspace*{6em}
	
	\begin{thm}\label{thm 2.5.4}
		\textbf{正则性}. \\
		If $E \subset \R$, the followings are equivalent:
		\begin{enumerate}
			\item[a.] $E \in \mathcal{M_\mu}$.
			
			\item[b.] $E = V \backslash N_1$ where $V$ is a $G_\delta$ set and $\mu(N_1) = 0$. 
			
			\item[c.] $E = H \cup N_2$ where $H$ is an $F_\sigma$ det and $\mu(N_2) = 0$.
		\end{enumerate}
	\end{thm}

	\vspace*{6em}
	
	\begin{proposition}\label{prop 2.5.2}
		If $E \in \mathcal{M}_\mu$ and $\mu(E) < \infty$, then $\forall \epsilon > 0$, $\exists A = $ finite disjoint union of open intervals, $\st$
		\begin{align}
			\mu(E \triangle A) < \epsilon
		\end{align}
	\end{proposition}
	
	\newpage
	
	\begin{thm}\label{thm 2.5.5}
		If $E \in \mathcal{L}$, then
		\begin{align}
			E + s \in \mathcal{L} \,\, and \,\, rE \in \mathcal{L} , \,\, \forall s , r \in \R
		\end{align}
		Moreover, 
		\begin{align}
			m(E + s) = m(E) \,\, and \,\, m(rE) = \left| r \right| m(E)
		\end{align}
	\end{thm}
	
	\vspace*{20em}
	
	\begin{proposition}\label{prop 2.5.3}
		Let $C$ be the Cantor set.
		
		\vspace*{1em}
		\begin{enumerate}
			\item[a.] $C$ is compact, nowhere dense, and totally disconnected.
			\begin{center}
				(i.e. the only connected subset of $C$ are single points)
			\end{center}
			Moreover, $C$ has no isolated points.
			
			\vspace*{1em}
			
			\item[b.] $m(C) = 0$. 
			
			\vspace*{1em}
			
			\item[c.] $card(C) = \aleph$.
		\end{enumerate}
	\end{proposition}
	
	\newpage
	
	\begin{figure}[thbp!]
		\centering
		\includegraphics[width=1.0\linewidth]{figure/1.5}
		\caption{Measure Theory}
		\label{pic : 4.7.2-1} % 添加图像引用标签
	\end{figure}

	%  ############################
	\ifx\allfiles\undefined
\end{document}
\fi